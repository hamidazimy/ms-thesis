\section{اهمیت موضوع و کاربردهای آن}

مسئلهٔ پیش‌بینی پیوند می‌تواند کاربردهای متفاوت و متنوعی در انواع مختلف شبکه‌ها داشته باشد. برای مثال، پیشنهاد کردن کالا به کاربر در وبگاه‌های خرید و فروش برخط نظیر آمازون\LTRfootnote{Amazon} یا ای‌بِی\LTRfootnote{eBay} می‌تواند به عنوان یک مسئلهٔ پیش‌بینی پیوند در شبکه‌های دوبخشی\LTRfootnote{Bipartite Network} کاربر-کالا در نظر گرفته شود و پیشنهادهای دقیق و مناسب می‌تواند فروش این وبگاه‌ها را به میزان قابل توجهی افزایش دهد. مسئلهٔ پیش‌بینی پیوند همچنین می‌تواند در حوزه‌های دیگری نظیر پیش‌بینی همکاری‌های آینده در شبکه‌های همکاری بین نویسندگان و مؤلفان\LTRfootnote{Co-authorship Network}، تشخیص همکاری‌های زیرزمینی بین تروریست‌ها و… استفاده شود. همچنین از پیش‌بینی پیوند می‌توان برای حل مسائل طبقه‌بندی در گراف‌هایی که به صورت ناقص برچسب‌گذاری شده‌اند\LTRfootnote{Partially Labeled}، برای تشخیص کارکرد پروتئین‌ها یا تشخیص ایمیل‌های ناخواسته استفاده کرد.

در بسیاری از شبکه‌های زیستی مثل زنجیرهٔ غذایی، شبکهٔ ارتباط میان پروتئین‌ها و شبکه‌های متابولیسمی، برای تشخیص وجود پیوند بین دو گره، می‌بایست آزمایش‌هایی در آزمایشگاه انجام شوند که این آزمایش‌ها معمولاً بسیار هزینه‌بر هستند. دانش ما از این نوع شبکه‌ها بسیار محدود است. برای مثال ۸۰٪ از تعاملات مولکولی در سلول‌های مخمر و ۹۹٫۷٪ در سلول‌های انسان، هنوز ناشناخته‌اند \cite{stumpf2008estimating} \cite{amaral2008truer}. به جای بررسی کردن تمام تعاملات ممکن، پیش‌بینی کردن بر اساس تعاملات شناخته‌شده و سپس تمرکز کردن بر روی پیوند‌هایی که با احتمال بیشتری وجود دارند، با فرض این که پیش‌بینی ما دقت خوبی داشته باشد، می‌تواند به مقدار قابل توجهی در هزینه‌های آزمایش صرفه‌جویی کند. تحلیل شبکه‌های اجتماعی نیز با موضوع داده‌های گم‌شده مواجه هستند\cite{neal2008kracking}، که در آن‌جا، الگوریتم‌های پیش‌بینی پیوند نقش مهمی ایفا می‌کنند. به علاوه، داده‌هایی که برای ساخت شبکه‌های زیستی یا اجتماعی استفاده می‌شود، ممکن است حاوی اطلاعات نادقیق باشد که این امر باعث می‌شود تا پیوند‌های جعلی ایجاد شوند\cite{von2002comparative}. مسئلهٔ پیش‌بینی پیوند می‌تواند برای تشخیص این پیوندهای جعلی نیز به کار گرفته شود\cite{guimera2009missing}.

علاوه بر این که الگوریتم‌های پیش‌بینی پیوند می‌توانند در یافتن داده‌های گم‌شده به ما کمک کنند، این الگوریتم‌ها می‌توانند برای پیش‌بینی پیوندهایی که ممکن است در آینده در یک شبکهٔ در حال تغییر و تحول ایجاد شوند نیز مورد استفاده قرار گیرند. برای مثال، در یک شبکهٔ اجتماعی برخط، پیوندهایی که در حال حاضر وجود ندارند اما شانس تشکیل‌شان بسیار بالاست، می‌توانند به عنوان دوستی‌های بالقوه به کاربران\LTRfootnote{Users} آن شبکهٔ اجتماعی پیشنهاد شوند، که همین موضوع می‌تواند کاربران را در یافتن دوست‌های جدید یاری کند و موجب تقویت وفاداری و افزایش میزان استفاده کاربران از شبکهٔ اجتماعی شود. مشابه همین روش می‌تواند برای ارزیابی سازوکار تغییر و تحول یک شبکهٔ مشخص مورد استفاده قرار بگیرد. برای مثال مدل‌های بسیاری برای تغییر و تحول ساختار شبکهٔ جهانی اینترنت ارائه شده است: بعضی از آن‌ها سعی می‌کنند به طور دقیق‌تر توزیع درجه‌ها و نحوهٔ اتصال گره‌ها به هم را بازتولید کنند\cite{zhou2004accurately}، برخی دیگر می‌کوشند ساختارهای $k$-هسته‌ای را بهتر توصیف کنند\cite{carmi2007model} و غیره. از آن‌جایی که ویژگی‌های ساختاری بسیاری برای شبکه‌ها وجود دارد و وزن‌دهی به آن‌ها بسیار سخت است، قضاوت این که کدام مدل (برای مثال کدام سازوکار تغییر و تحول) از بقیه بهتر است، آسان نیست. باید توجه داشت که هر مدل در اصل به یک الگوریتم پیش‌بینی پیوند متناظر می‌شود و بنابراین ما می‌توانیم از معیار دقت پیش‌بینی برای ارزیابی کارایی مدل‌های متفاوت استفاده کنیم.

همان‌طور که مشاهده می‌شود، مسئلهٔ پیش‌بینی پیوند، طیف بسیار وسیعی از کاربرها را شامل می‌شود. با توجه به این موضوع، لزوم یافتن روش‌هایی با کارایی بالا، بدیهی به نظر می‌رسد.
