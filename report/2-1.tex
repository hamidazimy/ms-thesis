\section{مقدمه}
در این بخش، ابتدا به تعریف مسئلهٔ پیش‌بینی پیوند پرداخته می‌شود و این مسئله به صورتی رسمی و ریاضی مدل خواهد شد. نخست به تعریف عمومی مسئله پرداخته می‌شود و سپس به حالت وزن‌دار که در این پژوهش مد نظر است اشاره خواهد شد. در ادامه روش‌های مختلف پیش‌بینی پیوند بررسی می‌شوند و دربارهٔ هر کدام توضیح مختصری بیان خواهد شد . دسته روش‌هایی که پایهٔ راهکار ارائه‌شده در این پژوهش است یعنی روش‌های مبتنی بر معیارهای شباهت محلی با جزییات بیشتری بررسی می‌شوند. در بخش بعد به توضیح مختصری در باب روش‌های \CommunityDetection پرداخته خواهد شد؛ چرا که در روش پیشنهادی که در بخش‌های آیندهٔ این پژوهش ارائه می‌شوند، از روش‌های \CommunityDetection بهره برده شده‌است.
\\[1cm]
%به طور مشخص، ورودی ما در بیان ریاضی این مسئله یک گراف ساده است که یال‌های آن می‌توانند جهت‌دار یا بی‌جهت باشد (بین هر دو گره حداکثر یک یال وجود دارد، و هیچ یالی وجود ندارد که یک گره را به خودش وصل کند). 
%خروجی مسئله، مجموعه‌ای از یال‌های بالقوه است که ممکن ناموجود و یا ناشناخته باشند و ما پیش‌بینی می‌کنیم که احتمال تشکیل و یا وجود آن‌ها بالا است.
