\newpage
\titleformat{name=\section,numberless}[display]{\Large\bfseries\Narges}{}{0pt}{\vspace{-1in}}[]

\section*{چکیده}
\setstretch{2.0}
\setlength{\parindent}{2em}
\setlength{\parskip}{1em}
امروزه تحلیل و بررسی شبکه‌های اجتماعی به موضوع مهمی تبدیل شده‌ و توجه  پژوهشگران رشته‌های مختلفی را برانگیخته است. در این میان یکی از مسائل مهم موجود، مسئلهٔ پیش‌بینی پیوند است. این مسئله می‌کوشد روابطی که هنوز در یک شبکه شناخته و یا تشکیل نشده‌اند را پیش‌بینی کند. برای حل این مسئله روش‌های بسیاری ارائه شده‌اند. یک دسته از روش‌هایی که برای حل این مسئله وجود دارد، شاخص‌های مبتنی بر شباهت ساختاری هستند که به علت سادگی و کارایی مناسب، محبوبیت زیادی در بین روش‌های پیش‌بینی پیوند دارند. از طرفی در بیشتر پژوهش‌های انجام شده در این زمینه، وزن پیوندها که نشان دهنده قدرت ارتباط است در نظر گرفته نشده است، در حالی که وزن ارتباطات حاوی اطلاعات مفیدی در این راستاست. همچنین می‌توان از اطلاعات ساختاری دیگری مانند انجمن‌های شبکه برای بهبود کارایی پیش‌بینی پیوند استفاده نمود.

هدف اصلی این پژوهش ارائه روشی بر پایهٔ تشخیص انجمن برای پیش‌بینی پیوند در شبکه‌های وزن‌دار است. به منظور تحقق این هدف، با در نظر گرفتن این نکته که احتمال تشکیل ارتباطات درون انجمن‌ها به نسبت بیشتر است، مسئلهٔ پیش‌بینی پیوند درون انجمن‌ها انجام شده‌است. راهکار پیشنهادی دو گام اساسی دارد که با توجه به استفاده یا عدم استفاده از وزن یال‌ها در هر دو گام، به چهار روش گسترش داده می‌شود. به منظور ارزیابی راهکار پیشنهادی، از شبکه‌های مصنوعی \lr{LFR} استفاده شده‌است که نوعی شبکه پارامتری مقیاس آزاد است. پس از انجام آزمایش روی فضای پارامتری این شبکه‌ها، تحلیلی از کارایی چهار روش پیشنهادی در هر بخش از فضای پارامتری ارائه شده و همچنین شرایطی که منجر به بهبود کارایی روش‌های پیش‌بینی پیوند می‌گردند بررسی می‌شوند.

{\Titr \textbf{\textit{واژه‌های کلیدی: }}}
{\small تحلیل شبکه‌های اجتماعی، پیش‌بینی پیوند وزن‌دار، تشخیص انجمن‌ها، شبکه‌های \lr{LFR}}

\newpage\null\thispagestyle{empty}\newpage
