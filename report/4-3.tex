\section{معیارهای ارزیابی}\label{subsec:evalu}
دو معیار استاندارد برای ارزیابی کارایی الگوریتم‌های پیش‌بینی پیوند استفاده می‌شود: \textit{دقت}\LTRfootnote{Precision} و \textit{\lr{AUC}}\footnote{سطح زیر نمودار منحنی \lr{ROC} یا به انگلیسی \lr{\textit{Area Under the receiver operating   characteristic Curve}} یا به اختصار \lr{AUC}}\cite{hanley1982meaning}. اساساً الگوریتم‌های پیش‌بینی پیوند، یک لیست مرتب از تمام پیوند‌های ناموجود به ما می‌دهند (معادل $U-E'$) یا به بیان دیگر، به هر پیوند ناموجود (فرض کنید $(x,y)\in U-E'$) یک امتیاز نسبت می‌دهد (فرض کنید $S_{xy}$) که با استفاده از آن، شانس وجود و یا تشکیل پیوند بین آن دو را کمّی کند. معیار \lr{AUC}، کارایی الگوریتم را با توجه به کل لیست ارزیابی می‌کند، در حالی که معیار دقت فقط روی $L$ پیوند ابتدایی لیست با بالاترین امتیاز تمرکز می‌کند. تعریف دقیق‌تر این دو معیار به صورت زیر است:
\begin{description}
  \item[معیار \lr{AUC}:]
  یک لیست مرتب از امتیاز تمام پیوندهای دیده‌نشده (در مجموعهٔ آزمایش) ساخته می‌شود. مقدار \lr{AUC} برابر است با احتمال این که یک پیوند گم‌شده (یعنی پیوندی که در مجموعهٔ آزمون قرار دارد $l_1 \in E''$) که به صورت تصادفی انتخاب شده، امتیاز بیشتری نسبت پیوند ناموجودی (یعنی پیوندی که در مجموعه یال‌های گراف اصلی وجود ندارد $l_2 \in U - E$) داشته باشد که آن نیز به صوردت تصادفی انتخاب شده است. یک پیاده‌سازی الگوریتمی برای این معیار به این صورت است: ابتدا برای هر پیوند دیده‌نشده، امتیاز گفته شده محاسبه می‌شود (با روش‌هایی که در ادامه بحث خواهیم کرد). در این حالت خوشبختانه نیازی به مرتب کردن لیست که کاری پرهزینه است نداریم. سپس هر بار یک پیوند از مجموعهٔ پیوندهای گم‌شده و یک پیوند از مجموعه پیوندهای ناموجود، هر دو به صورت تصادفی، انتخاب می‌کنیم امتیاز آن‌ها را مقایسه می‌کنیم. اگر از بین $n$ مقایسهٔ مستقل، در $n'$ حالت امتیاز پیوند گم‌شده از پیوند ناموجود بیشتر بود و در $n''$ حالت امتیازها برابر بود، مقدار \lr{AUC} برابر خواهد بود با:
\begin{equation}
  AUC = \frac{n' + 0.5n''}{n}
\end{equation}

اگر تمام امتیازهایی که به پیوندهای دیده‌نشده اختصاص داده‌ایم، متغیرهای تصادفی مستقل با توزیع یکسان\LTRfootnote{Independent and identically distributed (i.i.d.)} باشند، مقدار \lr{AUC} میبایست حدود $0.5$ به دست آید. بنابراین هر چقدر که این مقدار از $0.5$ بیشتر شود، نشان‌دهندهٔ این است که چه مقدار الگوریتم ما از شانس مطلق بهتر عمل می‌کند.
  \item[دقت:]
  با داشتن رتبه‌بندی پیوندهای دیده‌نشده، معیار دقت برابر است با نسبت پیوندهایی که درست پیش‌بینی شده‌اند به تعداد پیش‌بینی‌های انجام شده. این معیار به صورت زیر تعریف می‌شود:
\begin{equation}
  precision = \frac{TP}{TP+FP}
\end{equation}
که در آن،TP\LTRfootnote{\lr{true positive}} تعداد پیش‌بینی‌های درست و FP\LTRfootnote{\lr{false positive}} تعداد پیش‌بینی‌های غلط است. به عبارت دیگر، اگر ما به تعداد $N$ پیوند از ابتدای لیست پیش‌بینی‌ها برداریم که از این تعداد، $N_tp$ پیش‌بینی درست داشته باشیم (یعنی پیوند پیش‌بینی‌شده‌ای که در مجموعهٔ آزمون یا همان $E''$ وجود داشته باشد)، معیار دقت برابر خواهد بود با $N_tp / N$ که نام دیگر آن دقت $n$-بهترین\LTRfootnote{\lr{Top-n precision}} است. واضح است که دقت بیشتر یعنی کارایی بهتر الگوریتم.

علاوه بر این می‌توان برای محاسبهٔ دقت، مقدار $N$ یعنی تعداد پیش‌بینی‌ها را ثابت در نظر نگرفت و دقت را بر حسب تعداد پیش‌بینی‌ها در نموداری ترسیم کرد. این نمودار که دقت را بر حسب تعداد پیش‌بینی‌ها نشان می‌دهد، نمودار دقت-در-$n$\LTRfootnote{\lr{precision at n}} یا به اختصار $P@n$ خوانده می‌شود. منحنی‌هایی را که از رسم $P@n$ به ازای هر $n$ به دست می‌آید، می‌توان علاوه بر مقایسهٔ شهودی، با رابطهٔ میانگین دقت\LTRfootnote{\lr{Average Precision}} با هم مقایسه کرد که این رابطه به شکل زیر است:
\begin{equation}\label{eq:avgprecision}
  Average Presicion = \frac{\sum_{r = 1}^{n}{(r \times P@r)}}{\sum_{r = 1}^{n}{r}}
\end{equation} 
\end{description}

با استفاده از شکل \ref{fig:1} نحوهٔ محاسبهٔ معیارهای دقت و \lr{AUC} توضیح داده می‌شود. این گراف ساده، پنج گره، هفت پیوند موجود و سه پیوند ناموجود (پیوندهای $(1,2)$, $(1,4)$ و $(3,4)$) دارد. برای به دست آوردن دقت الگوریتم، ما باید تعدادی از پیوندهای موجود را به عنوان مجموعهٔ آزمون جدا کنیم. به عنوان مثال ما $(1,3)$ و $(4,5)$ را به عنوان مجموعهٔ آزمون انتخاب می‌کنیم که با نقطه‌چین در نمودار سمت چپ مشخص شده‌است. بنابراین هر الگوریتم می‌تواند فقط از اطلاعات مجموعهٔ آموزش استفاده کند که در نمودار سمت چپ با خط نشان داده شده‌اند. فرض کنید یک الگوریتم فرضی به تمام پیوندهای دیده نشده این امتیازها را بدهد: 
$s_{12}=0.4, s_{13}=0.5, s_{14}=0.6, s_{34}=0.5, s_{45}=0.6$.
برای محاسبهٔ \lr{AUC} ما می‌بایست امتیاز پیوندهای دیده نشده و ناموجود را مقایسه کنیم. در کل شش مقایسه وجود دارد:
$s_{13} > s_{12}, s_{13} < s_{14}, s_{13} = s_{34}, s_{45} > s_{12}, s_{45} = s_{14}, s_{45} > s_{34}$.
بنابراین مقدار AUC برابر خواهد بود با
$(3 \times 1 + 2 \times 0.5) / 6 \approx 0.67$.
برای محاسبهٔ دقت، اگه $L=2$ در نظر بگیریم، پیوندهای پیش‌بینی‌شده، $(1,4)$ و $(4,5)$ خواهند بود و به دلیل این که یکی درست و یکی غلط است، دقت پیش‌بینی برابر ۰٫۵ خواهد بود.
\begin{figure}[!hbt]
  \begin{subfigure}{.5\textwidth}
    \centering
    \begin{tikzpicture}
      \graph [color class=red, color class=green, math nodes, counterclockwise, n=5, nodes={circle, draw}] {
        { 1, 2, 3, 4, 5 },
        { 1 -- 3, 1 -- 5, 2 -- 3, 2 -- 4, 2 -- 5, 3 -- 5, 4 -- 5 }
      };
    \end{tikzpicture}
    \caption{شبکه اصلی}
  \end{subfigure}
  \begin{subfigure}{.5\textwidth}
    \centering
    \begin{tikzpicture}
      \graph [color class=red, color class=green, math nodes, counterclockwise, n=5, nodes={circle, draw}] {
        { 1, 2, 3, 4, 5 },
        { 1 --[dotted] 3, 1 -- 5, 2 -- 3, 2 -- 4, 2 -- 5, 3 -- 5, 4 --[dotted] 5 }
      };
    \end{tikzpicture}
    \caption{شبکه تقسیم شده به دو مجموعهٔ آموزش و آزمون}
  \end{subfigure}
  \caption{یک شبکهٔ ساده برای توضیح معیارهای ارزیابی}
  \label{fig:1}
\end{figure}
