\newpage
\titleformat{name=\section,numberless}[display]{\Large\bfseries\GentiumAlt}{}{0pt}{\vspace{-1in}}[]
\begin{latin}
\section*{Abstract}
\setstretch{2.0}
\setlength{\parindent}{2em}
\setlength{\parskip}{1em}
Nowadays, analysing social networks has became an important issue and it has attracted attentions from various fields of science. One of the most important problems here, is Link Prediction. This problem tries to predict the links that are either non-existent or unobserved. There are different approaches and methods toward this problem. Similarity-based methods is a category among them which is very popular due to its simplicity and resonable performance. Morever, in most of the previous works on this problem, link weights are not taken into account, even though they can carry valuable information. Similarly, one can use other structral information of a graph such as community information, to increase the performance of link prediction.

This study aims to propose a method based on community detection for link prediction in weighted networks. Briefly, the proposed method predict links inside communitie. The main reason for doing so is that its more likely for a node to establish a connection to a member of its own community, and also potetnial links inside of a community are much fewer than potential links outside of communities. This method consists of two steps and either involving or not involving the link weights in each of these steps, provide four different methods. For evaluating the performance of the proposed methods, a set of synthesis networks called LFR networks will be used which are kind of scale-free networks. After performing experiments on parameter space of these networks, we will analyze performance of the proposed methods and discuss that each of these methods can improve the performance of link prediction under what circumestances.

\textbf{\textit{Keywords: }}
{\small Social Network Analysis, Weighted Link Prediction, Community Detection, LFR Networks}
\end{latin}
\newpage\null\thispagestyle{empty}\newpage
