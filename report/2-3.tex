\section{مروری بر ادبیات پیش‌بینی پیوند}


مسئله‌ٔ پیش‌بینی پیوند یک چالش قدیمی در دانش اطلاعات مدرن است و الگوریتم‌های بسیاری در این زمینه ارائه شده‌اند. این الگوریتم‌ها طیف وسیعی را شامل می‌شوند که از روش‌های برپایهٔ شباهت گره‌ها گرفته تا روش‌های بر پایهٔ زنجیره‌های مارکوف\LTRfootnote{Markov chains} و مدل‌های آماری، گسترده شده‌اند. دسته‌بندی‌های مختلفی توسط افراد مختلف از این روش‌ها ارائه شده‌است. برای مثال لو\LTRfootnote{Lu} و ژو\LTRfootnote{Zhou}\cite{lu2011link} در سال ۲۰۱۱ این روش‌ها را به سه دستهٔ کلی زیر تقسیم کرده‌اند:
\begin{enumerate}
  %\onehalfspacing
  \item الگوریتم‌های بر پایهٔ شباهت\LTRfootnote{Similarity-based Algorithms} %\\[-4em]
  \item روش‌های \MaximumLikelihood\LTRfootnote{Maximum Likelihood} %\\[-4em]
  \item مدل‌های احتمالاتی\LTRfootnote{Probabilistic Models}
\end{enumerate}

یک دسته‌بندی دیگر از روش‌ها توسط ونگ\LTRfootnote{Wang} و همکاران\cite{wang2015link} در سال ۲۰۱۵ ارائه شد که روش‌ها را به چهار دستهٔ کلی زیر تقسیم‌بندی می‌کند:
\begin{enumerate}
  %\onehalfspacing
  \item معیارهای بر پایهٔ گره\LTRfootnote{Node-based Metrics}
  \item معیارهای بر پایهٔ \Topology\LTRfootnote{Topology-based Metrics}
  \item معیارهای بر پایهٔ نظریه اجتماعی\LTRfootnote{Social Theory based Metrics}
  \item روش‌های بر پایهٔ یادگیری\LTRfootnote{Learning-based Methods}
\end{enumerate}

در ادامه به توضیح مختصری دربارهٔ هر کدام از دسته روش‌ها پرداخته خواهد شد و روش‌هایی که مد نظر این پژوهش هستند به تفصیل مورد بررسی قرار خواهند گرفت.

% اما تحقبقات آن‌ها نمی‌توانند با پیشرفت‌های کنونی مطالعهٔ شبکه‌های پیشرفته رقابت کنند. به طور ویژه، این تحقیقات عمیقاً از در نظر نگرفتن ویژگی‌های ساختاری شبکه‌ها (مثل ساختار سلسه‌مراتبی [۲۱] و ساختار جوامع [۲۲]) که می‌توانند اطلاعات بسیار ارزشمند و دید مناسبی دربارهٔ پیش‌بینی پیوند به ارمغان آورند، رنج می‌برند. اخیراً رویکردهایی فیزیکی نظیر  پروسه‌های قدم‌زدن تصادفی\footnote{\lr{Random Walk}} و روش‌های \MaximumLikelihood\footnote{\lr{Maximum Likelihood methods}}، در مسئلهٔ پیش‌بینی پیوند کاربرد پیدا کرده اند… .

\subsection{معیارهای بر پایهٔ گره}
محاسبهٔ شباهت بین یک جفت گره راه حلی \intuitive برای پیش‌بینی پیوند است. این معیار بر پایهٔ ایده‌ای ساده‌ استوار است: «جفت گره‌ای \Likelihood وجود پیوند بین آن‌ها بیشتر است که بیشتر شبیه به هم هستند و بالعکس»\cite{wang2015link}. این ایده بر اساس این واقعیت است که کاربران به ایجاد روابط با افرادی که در آموزش، مذهب، علایق، مکان و… مشابه آن‌ها می‌باشند، تمایل دارند. تشابه این‌گونه اندازه‌گیری می‌شود که به هر جفت غیرمتصل از گره‌ها مثل $(x,y)$ نمرهٔ مفهوم شباهت بین آن دو تخصیص می‌شود. واضح است که نمرهٔ بالا نشان‌دهندهٔ احتمال بیشتر ایجاد پیوند بین $x$ و $y$ در آینده خواهد بود، و به طور مشخص  نمره کم نیز نشان می دهد که به احتمال زیاد دو گره $x$ و $y$ به یکدیگر متصل نخواهند شد. بنابراین، با استفاده از رتبهٔ نمرات تشابه بین گره‌ها، می‌توان تشکیل یا عدم تشکیل پیوندهایی در آینده و یا پیوند نهان در شبکه فعلی را پیش‌بینی کرد.
در یک شبکه اجتماعی \practical، یک گره معمولا دارای برخی ویژگی ها از قبیل مشخصات کاربری در شبکه‌های اجتماعی برخط، سابقه انتشار\LTRfootnote{Publication Record} در شبکه‌های اجتماعی دانشگاهی و… است. این اطلاعات می‌تواند به طور مستقیم برای محاسبهٔ شباهت بین دو گره استفاده شود. از آن‌جا که در بیشتر موارد، مقادیر ویژگی‌های گره‌ها به شکل متنی هستند، معیارهای شباهتی که معمولا مورد استفاده قرار می‌گیرند بر پایهٔ متن\LTRfootnote{Text-based} و بر پایهٔ رشته\LTRfootnote{String-based} هستند.
در مقالهٔ \cite{bhattacharyya2011analysis} یک مدل درختی طبقه‌بندی چندگانه تعریف شده که به مطالعهٔ کلمات کلیدی\LTRfootnote{Keywords} پروفایل کاربر می‌پردازد، سپس فاصلهٔ بین کلمات کلیدی را تعیین می‌کنند تا شباهت بین هر جفت از کاربران مشخص شود.
در نتیجه، معیارهای بر پایهٔ گره عمدتا از ویژگی‌ها فردی و \actions کاربران استفاده می‌کند، که می‌تواند منعکس‌کنندهٔ علایق شخصی و رفتارهای اجتماعی آنها برای محاسبه شباهت بین جفت گره‌ها باشد. بنابراین، معیارهای بر پایهٔ گره در پیش‌بینی پیوند مفید هستند به شرطی که بتوانیم ویژگی‌ها فردی و \actions کاربران را در شبکه‌های اجتماعی به دست آوریم\cite{anderson2012effects}.

\subsection{روش‌های بر پایهٔ شباهت}\label{subsec:similarity}
ساده‌ترین \Framework روش‌های پیش‌بینی پیوند، الگوریتم‌های بر پایه‌ٔ شباهت هستند. در بعضی منابع به این روش‌ها روش‌های بر پایهٔ همبندی\LTRfootnote{Topology-based} یا روش‌های بر پایهٔ شباهت ساختاری\LTRfootnote{Structural Similarity} نیز گفته می‌شود.
%در این روش‌ها به هر جفت از گره‌های $x$ و $y$، امتیازی مثل $S_{xy}$ تخصیص داده‌ می‌شود که به طور مستقیم شباهت بین $x$ و $y$ معنا می‌شود. % (یا در ادبیات \Proximity نامیده می‌شود). تمام پیوندهای دیده نشده با توجه به امتیازهای خود رتبه‌بندی می‌شوند و پیوند‌هایی که با گره‌های مشابه بیشتری در ارتباط هستند \Likelihood وجود بالاتری برایشان تصور می‌شود.

با وجود سادگی آن‌ها، مطالعه بر روی الگوریتم‌های برپایهٔ شباهت خود موضوع مهمی است. در حقیقت، تعریف شباهت گره یک چالش کوچک اما بااهمیت است. شاخص تشابه می‌تواند بسیار آسان یا بسیار پیچیده باشد. هر شاخص ممکن است برای برخی شبکه‌ها به خوبی پاسخگو باشد و اما در عین حال برای بعضی شبکه‌های دیگر با شکست مواجه شود.
%علاوه بر این، شباهت‌ها می‌توانند در راه‌های پیشرفته‌تری مانند به صورت محلی تحت فیلترهای مشترک\footnote{فیلترهای مشترک، فرآیندی از فیلتر برای اطلاعات یا الگوها با استفاده از روش‌هایی شامل اشتراک بین چندین عامل نظیر نظرات، منابع داده‌ها و غیره  است.} \Framework یکپارچه به کار گرفته شوند.
همان‌طور که در بخش پیش گفته شد، شباهت گره می‌تواند با استفاده از ویژگی‌های اساسی گره‌ها تعریف شود. دو گره مشابه در نظر گرفته می‌شوند اگر که ویژگی‌های مشترک زیادی با یکدیگر داشته باشند. اما از آن‌جایی که ویژگی‌های گره‌ها عموماً در دسترس نیستند، در نتیجه معمولاً بر گروه دیگری از شاخص‌های شباهت با نام شباهت ساختاری متمرکز می‌شوند که فقط بر پایهٔ ساختار شبکه استوار است.

شاخص‌های شباهت ساختاری را می‌توان به روش‌های متعددی طبقه‌بندی کرد، از جمله طبقه‌بندی‌ها می‌توان به محلی\LTRfootnote{Local} در مقابل سراسری\LTRfootnote{Global}، بدون پارامتر در قیاس با وابسته به پارامتر، وابسته به گره در مقابل وابسته به مسیر و غیره اشاره کرد.

%شاخص‌های شباهت همچنین می‌توانند از منظری دیگر به هم‌ارزی ساختاری و هم‌ارزی \Regular طبقه‌بندی شوند.روش پیشین یک فرض نهان  را متصور شده‌است، که پیوند، خود شباهت بین دو نقطهٔ نهایی را نشان می‌دهد (به عنوان نمونه به شاخص لنچ-هولم- نیوتون (۳۲)و انتقال شباهت (۳۳) نگاه کنید)؟؟؟ در حالی که در دومی فرض می‌شود که دو گره مشابه هستند اگر همسایه‌های آن‌ها مشابه باشند.
%به خوانندگان توصیه می‌شود که مرجع شماره ۳۴ را برای تعریف ریاضی هم‌ارزی باقاعده و مرجع ۳۵ را برای کاربرد جدید پیش‌بینی توابع پروتیین مورد مطالعه قرار گیرد.؟؟؟
%در این‌جا ساده‌ترین روش انتخاب شده‌است،  که در آن ۲۰ شاخص شباهت به ۳ دسته طبقه‌بندی می‌شوند که شامل ۱۰ شاخص محلی و ۷ شاخص سراسری و نیز ۳ شاخص \QuasiLocal است که به اطلاعات \Topology سراسری نیاز ندارد اما اطلاعات بیشتری  را نسبت به شاخص‌های محلی  به کار میبرد.؟؟
\subsubsection{شاخص‌های شباهت محلی}
این روش‌ها، زیرشاخه‌ای از روش‌های بر پایهٔ شباهت هستند که از اطلاعات محلی، یعنی اطلاعات خود گره‌ها و همسایگان آن‌ها استفاده می‌کنند. این روش‌ها به دلیل سادگی مفهومی و محاسباتی و همچنین کارایی مناسب، از محبوبیت بسیاری برخوردارند. پژوهش حاضر نیز از همین دسته از روش‌ها استفاده خواهد کرد. در زیر به معرفی این روش‌ها و توضیح جزییات آن‌ها پرداخته می‌شود.
\begin{description}
\item[همسایگان مشترک \lr{(CN)}]\LTRfootnote{Common Neighbors}\textbf{:}
%\textbf{همسایه‌های مشترک ($CN$)}
شاخص همسایگان مشترک به دلیل سادگی یکی از گسترده‌ترین شاخص‌های مورد استفاده در مسائل پیش‌بینی پیوند است. برای هر دو گره $x$ و $y$، این شاخص معرف تعداد گره‌هایی است که با هر دو گره $x$ و $y$ ارتباط مستقیمی داشته باشد و در واقع همسایهٔ مشترک هر دو گره باشند. بنابراین، دو گره $x$ و $y$ چنانچه همسایه‌های مشترک بسیاری داشته باشند با احتمال خوبی با یکدیگر نیز پیوند دارند. ساده‌ترین اندازه‌گیری این همپوشانی همسایه‌ها، شمارش مستقیم است.
برای یک گره با نام $x$،مجموعه‌ای از همسایه‌های $x$ با $\Gamma{(x)}$ نشان داده می‌شود:
\begin{equation}\label{eq:cn}
S_{xy}^{CN} = |\Gamma{x} \cap \Gamma{y}|
\end{equation}
که $|Q|$ تعداد اعضای مجموعهٔ $Q$ است. بدیهی است که 
$S_{xy}=(A^2)_{xy}$،
که در آن $A$ ماتریس مجاورت\LTRfootnote{Adjacency Matrix} است و چنانچه $x$ و $y$ به طور مستقیم با یکدیگر در ارتباط باشند و پیوندی بین آن‌ها باشد $A_{xy}=1$ و در غیر اینصورت $A_{xy}=0$ می‌باشد.توجه شود که همچنین $(A^2)_{xy}$ بیانگر تعداد مسیرهای مختلف به طول ۲ است که $x$ و $y$ را به هم متصل می‌کنند.
نیومن\LTRfootnote{Newman}\cite{newman2001clustering} این کمیت را در مطالعات خود در زمینهٔ \CollaborationNetworks مورد بررسی قرار داد و نشان داد همبستگی مثبتی بین تعداد همسایگان مشترک و احتمال این که دو محقق در آینده همکاری داشته باشند، وجود دارد.
کاسینتز\LTRfootnote{Kossinets} و واتس\LTRfootnote{Watts}\cite{kossinets2006effects} یک شبکه‌های اجتماعی با مقیاس بزرگ را تجزیه و تحلیل کردند که نشان می‌دهد دو دانشجو که دارای دوستان مشترک بسیاری هستند، دوست شدن آن‌ها در آینده از احتمال خوبی برخوردار است.
از آن‌جا که در این پژوهش هدف استفاده از وزن‌یال‌هاست، نیاز داریم که از گسترش وزن‌دار شاخص‌ها استفاده کنیم. موراتا\LTRfootnote{Murata} و موریاسو\LTRfootnote{Moriyasu} در سال ۲۰۰۷ گسترش وزن‌داری از سه شاخص شباهت ارائه دادند \cite{murata2007link}. اولین شاخص، شاخص همسایگان مشترک است که رابطهٔ گسترش وزن‌دار آن به صورت زیر است. دو شاخص دیگر در ادامه بررسی خواهند شد.
\begin{equation}
S_{xy}^{WCN} = \frac{1}{2}\sum_{z\in\Gamma(x)\cup\Gamma(y)}{w(x,z)+w(z,y)}
\end{equation}

از آن‌جا که شاخص همسایگان مشترک نرمال‌شده\LTRfootnote{Normalized} نیست، معمولا شباهت نسبی بین جفت گره‌ها را نشان می‌دهد. بنابراین، برخی از معیارهای دیگر بر پایهٔ همسایه‌ها، بررسی می‌کنند که چگونه می‌توان این معیار را به شکل منطقی نرمال کرد.

\item[شاخص سالتون]\LTRfootnote{Salton Index}\textbf{:}
%\textbf{شاخص سالتون}
این شاخص یک شاخص کوسینوسی برای محاسبهٔ شباهت بین دو گره $x$ و $y$ است و به شکل زیر تعریف می‌شود \cite{salton1986introduction}:
\begin{equation}
S^{Salton}_{xy}=\frac{|\Gamma(x)\cap\Gamma(y)|}{\sqrt{k_{x}\times k_{y}}}
\end{equation}
در این‌جا $k_{x}$ نشان‌دهندهٔ درجهٔ گره $x$ است. شاخص سالتون در بعضی مواقع \CosineSimilarity نیز نامیده می‌شود.

\item[شاخص جاکارد]\LTRfootnote{Jaccard Index}\textbf{:}
%\textbf{شاخص جاکارد}
این شاخص که اندازه همسایگان مشترک را نرمال می‌کند توسط جاکارد بالغ بر ۱۰۰ سال پیش ارائه شد. این شاخص فرض می‌کند که شباهت بیشتر، برای زوج گره‌هایی است که نسبت بالاتری از مجموع همسایگانشان بین آن‌ها مشترک است و به شکل زیر تعریف می‌شود:
\begin{equation}
S^{Jaccard}_{xy}=\frac{|\Gamma(x)\cap\Gamma(y)|}{|\Gamma(x)\cup\Gamma(y)|}
\end{equation}

\item[شاخص سورنسن]\LTRfootnote{Sorensen Index}\textbf{:}
%\textbf{شاخص سورنسن}
این شاخص علاوه بر توجه به اندازهٔ همسایه‌های مشترک، همچنین بیان می‌کند که گره‌هایی با مجموع درجهٔ پایین‌تر \Likelihood ایجاد پیوند بالاتری دارند. این شاخص عمدتاً برای داده‌های مربوط به محیط زیست مورد استفاده قرار می‌گرفته‌است و به صورت زیر تعریف می‌شود \cite{sorensen1948method}:
\begin{equation}
S_{xy}^{Sorensen}=\frac{2|\Gamma(x)\cap\Gamma(y)|}{k_{x}+k_{y}}
\end{equation}

\item[شاخص \lr{HP}]\LTRfootnote{Hub Promoted Index}\textbf{:}
%\textbf{شاخص \lr{HPI}}
این شاخص همپوشانی \topological\LTRfootnote{Topological} بین دو گره را محاسبه می‌کند و برای استفاده در شبکه‌های زیستی پیشنهاد شده است. این شاخص به صورت زیر تعریف می‌شود:
\begin{equation}
S_{xy}^{HPI}=\frac{|\Gamma(x)\cap\Gamma(y)|}{\min\{ k_{x},k_{y} \}}
\end{equation}

بر اساس این تعریف، پیوندهای مجاور به \Hubs محتمل‌تر هستند که نمرات بالاتری به آن‌ها تخصیص یابد به این دلیل که درجهٔ کوچک‌تر، مخرج را تعیین می‌کند. به بیانی دیگر، ارزش این شاخص توسط گره‌های با درجه کمتر تعیین می‌شود \cite{ravasz2002hierarchical}.

\item[شاخص \lr{HD}]\LTRfootnote{Hub Depressed Index}\textbf{:}
%\textbf{شاخص \lr{HDI}}
ژو و همکاران پیشنهاد یک شاخص مشابه \lr{HPI} را مطرح کردند \cite{zhou2009predicting}، اما ارزش را گره‌های با درجهٔ بالاتر تعیین می‌کنند. این شاخص به صورت زیر تعریف می‌شود:
\begin{equation}
S_{xy}^{HDI}=\frac{|\Gamma(x)\cap\Gamma(y)|}{\max\{ k_{x},k_{y} \}}
\end{equation}

\item[شاخص \lr{LHN}]\LTRfootnote{Leicht-Holme-Newman}\textbf{:}
%\textbf{شاخص \lr{LHN}}
شاخص لایت-هولم-نیومن یا \lr{LHN}، شباهت بیشتر را به زوج گره‌هایی که همسایگان بیشتری (نه در قیاس با بیشینه همسایگان محتمل بلکه) در مقایسه با تعداد همسایگان مورد انتظار دارند، تخصیص می‌دهد \cite{leicht2006vertex} و به صورت زیر تعریف می‌شود:
 \begin{equation}
S_{xy}^{LHN1}=\frac{|\Gamma(x)\cap\Gamma(y)|}{k_{x}\times k_{y}}
\end{equation}
همان‌طور که مشاهده می‌شود مخرج کسر بالا ($k_{x} \times k_{y}$) متناسب با تعداد همسایگان مورد انتظار مشترک گره‌های $x$ و $y$ است.

\item[شاخص وابسته به پارامتر \lr{(PD)}]\LTRfootnote{Parameter Dependent}\textbf{:}
%\textbf{شاخص \lr{PD}}
به منظور بهبود دقت برای پیش‌بینی هر دو دستهٔ پیوندهای محبوب و غیرمحبوب، ژو و همکاران معیار $PD$ را به شرح زیر پیشنهاد کردند \cite{zhu2012uncovering}. در این‌جا $\lambda$ یک پارامتر آزاد است. زمانی که $\lambda=0$ باشد، معیار \lr{PD} همان معیار همسایگان مشترک (\lr{CN}) است و اگر $\lambda=0.5$ و یا $\lambda=1$ باشد به ترتیب معادل معیارهای سالتون و \lr{LHN} می‌شود. 
\begin{equation}
S_{xy}^{PD}=\frac{|\Gamma(x)\cap\Gamma(y)|}{(|k_{x}|\times |k_{y}|)^\lambda}
\end{equation}

\item[شاخص وابستگی ترجیحی \lr{(PA)}]\LTRfootnote{Prefrential Attachment}\textbf{:}
%\textbf{شاخص \lr{PA}}
این شاخص نشان می‌دهد که پیوندهای جدید، بیشتر احتمال دارد به گره‌هایی با درجه بالاتر متصل شوند. این شاخص می‌تواند به منظور ایجاد تغییر و تحول در شبکه‌های \ScaleFree به کار گرفته شود، که در آن احتمال این که یک پیوند جدید به گره $x$ متصل شود متناسب با $k_{x}$ است. همچنین می‌توان از راهکار مشابهی در شبکه‌های \ScaleFree بدون رشد استفاده کرد که در آن در هر مرحله، یک پیوند قدیمی حذف شده‌ و یک پیوند جدید تولید می‌شود و احتمال آنکه پیوند جدید دو گره $x$ و $y$ را به یکدیگر متصل کند متناسب با $k_{x} \times k_{y}$ است \cite{newman2001clustering}.
\begin{equation}
S_{xy}^{PA}=k_{x}\times k_{y}
\end{equation}
باید توجه داشت که این معیار به اطلاعات همسایگان هر گره نیازی ندارد، در نتیجه از حداقل پیچیدگی محاسباتی برخوردار است. این شاخص یکی دیگر از شاخص‌هاییست که توسط موراتا و موریاسو به حالت وزن‌دار گسترش داده شد. گسترش وزن‌دار این معیار نیز به صورت زیر است:
\begin{equation}
S_{xy}^{WPA} = s(x) \times s(y)
\end{equation}
که در آن، $s(x)$ \Strength\LTRfootnote{strength} گره $x$ را مشخص می‌کند که عبارتست از مجموع وزن یال‌های متصل به گره $x$ و یا به صورت ریاضی
$s(x) = \sum_{i\in\Gamma(x)}{w(x,i)}$.

\item[شاخص آدامیک/ادار \lr{(AA)}]\LTRfootnote{Adamic/Adar}\textbf{:}
%\textbf{شاخص \lr{AA}}
این شاخص با شمارش سادهٔ همسایه‌های مشترک با اختصاص وزن بیشتر به همسایگانی که خود آن‌ها دارای همسایگان کمتری هستند، تعریف می‌شود. این معیار توسط آدامیک\LTRfootnote{Adamic} و ادار\LTRfootnote{Adar} در ابتدا برای محاسبهٔ شباهت بین دو صفحه وب پیشنهاد شد \cite{adamic2003friends}، که پس از آن به طور گسترده‌ای در شبکه‌های اجتماعی مورد استفاده قرار گرفت. رابطهٔ محاسبهٔ معیار \lr{AA} به شکل زیر تعریف شده‌است:
\begin{equation}
S_{xy}^{AA}= \sum_{z\in\Gamma(x)\cap\Gamma(y)}{\frac{1}{\log k_{z}}}.
\end{equation}
سومین شاخصی که گسترش وزن‌دار آن توسط موراتا و موریاسو معرفی شد، شاخص \lr{AA} است که به صورت زیر به نسخه وزن‌دار گسترش داده می‌شود:
\begin{equation}
 S_{xy}^{WAA} = \frac{1}{2}\sum_{z\in\Gamma(x)\cup\Gamma(y)}\frac{w(x,z)+w(z,y)}{log(1+s(z))}
\end{equation}
که جمع کردن عدد یک در مخرج به دلیل پرهیز از منفی شدن امتیازها برای مواقعی‌ست که وزن‌ها از یک کوچک‌ترند.

\item[شاخص تخصیص منابع \lr{(RA)}]\LTRfootnote{Resource Allocation}\textbf{:}
%\textbf{شاخص \lr{RA}}
معیار تخصیص منابع یا به اختصار \lr{RA} توسط ژو و همکاران ارائه شده‌است \cite{zhou2009predicting}. این معیار از فرآیندهای فیزیکی تخصیص منابع در شبکه‌های پیچیده الهام گرفته شده‌است. یک جفت گره را در نظر بگیرید، گره $x$ و $y$، که به صورت مستقیم به یکدیگر مرتبط نیستند. گره $x$ می‌تواند تعدادی منبع را به گره $y$ به وسیلهٔ همسایه‌های مشترک آن دو که نقش انتقال‌دهنده را ایفا می‌کنند، ارسال کند. در ساده‌ترین حالت، فرض می‌کنیم که هر انتقال‌دهنده دارای یک واحد از منابع است و به یک اندازه آن را بین تمام همسایگان خود توزیع می‌کند. شباهت بین $x$ و $y$ می‌تواند به عنوان مقدار منابعی که $y$ از $x$ دریافت می‌کند تعریف شود:
\begin{equation}
S_{xy}^{RA}= \sum_{z\in\Gamma(x)\cap\Gamma(y)}{\frac{1}{k_{z}}}.
\end{equation}
واضح است که این اندازه‌گیری متقارن است یعنی $S_{yx} = S_{xy}$. معیار \lr{RA} مشابه \lr{AA} است، هر دو معیار سهم تاثیر همسایگان مشترک با درجهٔ بالا را کاهش می‌دهند. با این وجود معیار \lr{RA} به نسبت \lr{AA}، سهم همسایگان مشترک درجهٔ بالا را شدیدتر می‌کاهد. معیار \lr{AA} فرمی اینگونه دارد $(\log k_{z})^{-1}$ در حالی که معیار \lr{RA} فرمی به صورت $k_{z}^{-1}$ دارد. بنابراین، \lr{RA} و \lr{AA} برای شبکه‌هایی که میانگین درجه گره‌های آن‌ها کم است، نتایج پیش‌بینی بسیار نزدیکی دارند در صورتی که \lr{RA}  برای شبکه‌هایی با میانگین درجهٔ بالا بهتر عمل می‌کند. علاوه بر این، \lr{RA} و \lr{AA} نه تنها از همسایگان مستقیم استفاده می‌کنند بلکه همسایگان همسایگان را نیز در نظر می‌گیرد و همین امر این دو شاخص را از سایر شاخص‌ها متمایز می‌کند.
این شاخص نیز توسط لو و ژو در سال ۲۰۱۰ با پیروی از نحوهٔ گسترشی که موراتا و موریاسو ارائه کرده بودند، به حالت وزن‌دار گسترش یافت \cite{lu2010link} که گسترش‌یافتهٔ آن به شکل زیر است:
\begin{equation}
S_{xy}^{WRA}(x,y) = \sum_{z\in\Gamma(x)\cup\Gamma(y)}\frac{w(x,z)+w(z,y)}{s(z)}
\end{equation}

\end{description}
به این دلیل که همسایگان می‌توانند به طور غیرمستقیم رفتار اجتماعی کاربران را منعکس کنند و به صورت مستقیم بر انتخاب اجتماعی کاربران تاثیرگذار هستند، بسیاری از روش‌های پیش‌بینی پیوند بر پایهٔ همسایه‌ها استوار است.

%For example, Akcora et al. propose
%a new global network similarity [4], in which they first define the mutual friends graph (MFG) of x and
%y. Edge count of the MFG can measure the strength of ties between x and y. To normalize the similarity
%value, they also define the friendship graph (FG) of node x. The network similarity can be measured
%through a comparison between the number of edges in MFG and the number of edges in FG.
%Sarkar et al. presented a theoretical justification of popular link prediction heuristics, to obtain com-
%mon empirical observations for neighbor-based metrics [83]. They justified that the number of common
%neighbors gives bounds on similarity of a node pair, therefore, some metrics would predict links with
%maximum number of common neighbors. They also presented theoretical justification for that metrics
%carefully using weighted count of common neighbors that often outperform the unweighted count. There-
%fore, simple metrics of counting common neighbors often outperform more complicated link prediction
%methods.
%
% O(n)??? & Tables???
%
\subsubsection{شاخص‌های بر پایهٔ مسیر}
یکی دیگر از زیرمجموعه‌های روش‌های بر پایهٔ شباهت، شاخص‌های بر پایهٔ مسیر هستند که بر خلاف دستهٔ پیش، فقط از اطلاعات محلی بهره نمی‌گیرند، بلکه علاوه بر آن، اطلاعات مسیرهای بین دو گره را نیز مورد استفاده قرار می‌دهند. در زیر به معرفی تعدادی از این شاخص‌ها پرداخته می‌شود.
\begin{description}
\item[مسیر محلی]\LTRfootnote{Local Path}\textbf{:}
%\textbf{مسیر محلی}
شاخص مسیر محلی یا به اختصار \lr{LP} از اطلاعات مسیرهای محلی با طول ۲ و ۳ استفاده می‌کند. برخلاف شاخص‌هایی که تنها اطلاعات نزدیک‌ترین همسایه‌ها را به کار می‌برند، این شاخص برخی از اطلاعات اضافی همسایه‌ها به فاصله‌ای با طول ۳ تا گره فعلی را مورد استفاده قرار می‌دهد. بدیهی است که مسیرهای با طول ۲ از مسیرهای با طول ۳ مناسب‌تر هستند \cite{sarkar2011theoretical}. بنابراین یک ضریب تنظیم\LTRfootnote{Adjustment Factor} $\alpha $ برای کاهش اثر مسیرهای با طول ۳ وجود دارد. مقدار $\alpha$ باید عددی کوچک و نزدیک به صفر باشد. این شاخص به شکل زیر تعریف شده‌است. در این‌جا، $A^{2}$ و $A^{3}$ نشان‌دهندهٔ ماتریس مجاورت گره‌هایی است که مسیر با طول ۲ و ۳ را دارا هستند. در نتیجه، \lr{LP} نیز ماتریس مجاورتی است که جفت گره‌ها با فاصله‌های به طول ۲ و ۳ را توصیف می‌کند \cite{lu2009similarity}.
\begin{equation}
LP = A^{2}+\alpha A^{3}
\end{equation}
\item[کتز]\LTRfootnote{Katz}\textbf{:}
%\textbf{کتز}
شاخص کتز بر پایهٔ ترکیبی از تمام مسیرهای بین دو گره است. در این روش از تمام مسیرهای بین دو گره استفاده می‌شود. شاخص کتز تعداد و طول تمام مسیرهای موجود بین دو بین گره $x$ و $y$ را به منظور پیش‌بینی پیوند به کار می‌گیرد \cite{katz1953new}. میرا شدن نمایی تاثیر مسیرها توسط طول آن‌ها می‌تواند وزن بیشتری را به مسیرهای کوتاه‌تر بدهد. این اندازه‌گیری به شرح زیر است، که در آن $path^{l}_{xy}$ که مجموعه‌ای از تمام مسیرها از $x$ به $y$ است با طول $l$ است و $\beta > 0$. مقدار بسیار کوچک $\beta$ باعث می‌شود که کتز بسیار شبیه به شاخص \lr{CN} شود، چرا که در این صورت، مسیرهای طولانی در شباهت نهایی، تاثیر بسیار کمی دارند.
\begin{equation}
Katz(x,y)=\sum^{\infty}_{l=1}\beta^{l}.|path^{l}_{x,y}| = \beta A + \beta^{2}A{2} +\beta^{3}A{3} + ...
\end{equation}

%\textbf{$RSS$} page 10 (paper 2015)
\item[پیوند دوستان]\LTRfootnote{FriendLink}\textbf{:}
%\textbf{پیوند دوستان}
شاخص پیوند دوستان یا به اختصار \lr{FL} محاسبهٔ شباهت بین گره $x$ و $y$ با پیمودن تمام مسیرهای با طول محدود به یک کران مشخص است. این شاخص می‌تواند (به دلیل در نظر گرفتن کران) پیش‌بینی پیوند دقیق‌تر و سریع‌تری را ارائه کند. پیوند دوستان فرض می‌کند که افراد در یک شبکهٔ اجتماعی می‌توانند تمام مسیرهای بین خود را به نسبت  طول مسیر به کار گیرند \cite{papadimitriou2012fast}. شباهت بین گره $x$ و $y$ به عنوان تعداد مسیرهای با طول مختلف $l$ از $x$ تا $y$ به شکل  زیر تعریف می‌شود : 
\begin{equation}
FL(x,y) = \sum^{l}_{i=1} \frac{1}{i-1}.\frac{|paths^{i}_{x,y}|}{\prod^{i}_{j=2}(n-j)}
\end{equation}
 
 که در آن $n$ تعداد گره‌های موجود در شبکه است، و $l$ طول مسیر بین $x$ و $y$ است (به استثنای مسیر با دور)، و $paths^{i}_{x,y}$ مجموعه‌ای از تمام مسیرهای از گره $x$ تا $y$ با طول $i$ است. با این حال، این به آن معنا نیست که $l$های بالاتر دقت عمل بیشتری را موجب می‌شوند. در واقع، با زیاد شدن بیش از حد $l$، دقت به مرور افت خواهد کرد.
%textbf{$VCP$} page 11 (paper 2015)
\end{description}
\subsubsection{شاخص‌های بر پایه ولگشت}
این روش‌ها می‌کوشند تا شباهت بین دو گره را با استفاده از ولگشت\LTRfootnote{Random Walk} (یا قدم زدن تصادفی) به دست آورند؛ به این صورت که از احتمال رفتن از یک گره به همسایه‌های آن در ولگشت استفاده می‌کنند و به این صورت معیاری شبیه فاصله بین گره‌ها به دست می‌آورند. در زیر به تعدادی از مهمترین شاخص‌های این دسته پرداخته خواهد شد.
\begin{description}
\item[سیم‌رنک]\LTRfootnote{SimRank}\textbf{:}
%\textbf{سیم‌رنک}
شاخص سیم‌رنک، با این فرض تعریف شده که دو گره مشابه هستند، اگر به گره‌های مشابه متصل باشند. $\gamma$  پارامتری است که کنترل می‌کند با چه سرعتی وزن گره‌های متصل به هم هنگامی که از گره اصلی  دور می‌شوند، کاهش یابد \cite{mcpherson2001birds}. %به طریق \selfconsistent تعریف شده‌است، با توجه به فرض،
\begin{equation}
  simRank(x, y)=  \left\{
              \begin{array}{lcl}
                1 & \text{if} &  x = y ,\\
                \gamma . \frac{\sum_{a\in \varGamma(x)} \sum_{b\in \varGamma(y)}simRank(a,b)}{|\varGamma(x)|.|\varGamma(y)|} & \text{if} & otherwise.
              \end{array} \right.
\end{equation}

سیم‌رنک را می توان با استفاده از مدل پیمایشگر جفت تصادفی\LTRfootnote{random surfer-pairs model} توضیح داد: $simRank(x, y)$ تعیین می‌کند که انتظار می‌رود دو پیمایشگر تصادفی که حرکت خود را از دو گره $x$ و $y$  آغاز کرده‌اند چقدر زود در یک گره یکدیگر را ملاقات کنند.
پیچیدگی زمانی سیم‌رنک $O(n^4)$ است که $n$ نمایانگر تعداد گرهٔ شبکه است \cite{jeh2002simrank}. به دلیل همین پیچیدگی زمانی بالا، این شاخص برای استفاده در شبکه‌های با مقیاس بزرگ مناسب نیست. 
 %اما تلاش‌هایی برای بهبود سرعت آن انجام شده‌است.

\item[رتبه‌صفحه ریشه‌دار]\LTRfootnote{Rooted PageRank}\textbf{:}
%\textbf{رتبه‌صفحه ریشه‌دار}
این شاخص نسخه اصلاح‌شدهٔ الگوریتم رتبه‌صفحه\LTRfootnote{PageRank} است که در موتورهای جستجو به منظور رتبه‌بندی صفحات وب به کار می‌رود \cite{liben2007link}. در این شاخص به هر گره، احتمالی تخصیص داده می‌شود که نشان‌دهندهٔ احتمال رسیدن یک ولگشت به آن گره است. عامل $\epsilon $ ضریبی است که مشخص می‌کند که چقدر امکان دارد ولگشت به جای برگشت به گره مبدا، گره همسایه را ملاقات کند.
 %هر چه مقدا۶ر $\epsilon $ بیشتر با۶شد، پیش‌بینی‌های انجام شده بیشتر به گره‌های پیرامون هر گره متمایل می‌شود. 
 \begin{equation}
RPR = (1-\epsilon)(I-\epsilon D^{-1} A^{-1})^{-1},\ \ \ \ \ \ D_{i,i} =  \sum_{j} A_{i,j}
\end{equation}
که $I$ ماتریس واحد و $A$ ماتریس مجاورت گراف است و $A_{i,i}$ زمانی یک می‌شود که بین دو گره $i$ و $j$ یالی موجود باشد، در غیر اینصورت مقدارش صفر است.

\item[پراپ‌فلو]\LTRfootnote{PropFlow}\textbf{:}
%\textbf{پراپ‌فلو}
%من واک رو گام و پیاده‌روی ترجمه کردم اینجا تو این پراپ‌فلو دیدی! کلن یه دور اینو با دقت بخون و اصلاح کن! هستی نوشته بود شرط توقفش سه علته! اما خب من اینجوری نوشتم (ترجمه کردم) باز یه مقایسه کن جا نیوفته 
این شاخص مشابه رتبه‌صفحه ریشه‌دار است، اما محلی‌سازی بیشتری روی آن انجام شده‌است. پراپ‌فلو متناسب با احتمال رسیدن یک ولگشت کراندار به $y$ است، که از $x$ شروع می‌شود و از $l$ گام هم بیشتر نیست. این ولگشت پیوندها را بر اساس وزن انتخاب می‌کند و زمانی که به گره $y$ برسد و یا این که مجدداً گره $x$ را ملاقات کند، خاتمه می‌یابد. این روش، عددی را تولید می‌کند که می‌توان به عنوان برآورد احتمال پیوندهای جدید به کار گرفته شود \cite{lichtenwalter2010new}. 
اگر $x$ و $y$ به طور مستقیم به یکدیگر پیوند داشته‌باشند، پراپ‌فلوی آن‌ها به صورت زیر محاسبه می‌شود:
\begin{equation}
PF(x,y) = PF(a,x)\frac{w_{xy}}{\sum_{k \in \varGamma(x)} w_{xk}}
\end{equation}
که در آن $k$ همسایهٔ گره $x$ است که عمق آن از نقطه شروع بیشتر از عمق گره $x$ است. $w_{xy}$ بیانگر وزن پیوند بین $x$ و $y$ است و $a$ گرهٔ پیشین $x$ در مسیر ولگشت است. اگر گرهٔ آغازین $x$ باشد، آنگاه: $PF(a, x) = 1$.
اگر $x$ و $y$ به طور غیرمستقیم با هم مرتبط باشند، $PF(x, y)$ مجموع پراپ‌فلوهای تمام کوتاه‌ترین مسیرها از $x$ به $y$ است.
برخلاف رتبه‌صفحه ریشه‌دار، محاسبهٔ پراپ‌فلو به راه‌اندازی مجدد ولگشت و یا همگرایی نیازی ندارد و در عوض به سادگی از یک جستجوی اول سطح محدود شده به ارتفاع $l$ استفاده می‌کند. بنابراین، راهکاری سریع‌تر از سیم‌رنک و رتبه‌صفحه ریشه‌دار است.
\end{description}
\subsection{روش‌های \MaximumLikelihood}
یک دسته از روش‌های پیش‌بینی پیوند، دسته روش‌های مبتنی بر تخمین \MaximumLikelihood هستند. این روش‌ها ابتدا یک سری اصول ساختاری برای ساختار شبکه در نظر می‌گیرند و با بیشینه کردن شباهت ساختار دیده شده از شبکه، مجموعه قوانین و پارامترهای مشخصی را به دست می‌آورند. سپس، احتمال وجود هر کدام از پیوندهای دیده‌نشده می‌تواند با توجه به این قوانین و پارامترها محاسبه شود.

از نقطه نظر کاربردهای تجربی، یک مشکل اساسی برای روش‌های \MaximumLikelihood این است که بسیار زمان‌بر هستند. یک الگوریتم خوب طراحی شده از این نوع، قادر است با شبکه‌هایی با حداکثر چند هزار گره در زمان قابل قبول کار کند، اما وقتی با گراف‌های بسیار بزرگ شبکه‌های اجتماعی برخط که معمولاً از میلیون‌ها گره تشکیل شده‌اند مواجه می‌شود، قادر به انجام عملیات نخواهد بود. علاوه بر این، روش‌های \MaximumLikelihood معمولاً جزو دقیق‌ترین و بهترین روش‌های پیش‌بینی پیوند نیستند. با این حال، این روش‌ها دید ارزشمندی از ساختار شبکه برای ما به ارمغان می‌آورند که در روش‌های دیگر مثل روش‌های بر پایهٔ شباهت یا روش‌های احتمالاتی نخواهیم داشت.

\subsection{مدل‌های احتمالاتی}
هدف روش‌های احتمالاتی این است که ساختار زیرین یک شبکه را استخراج کند و سپس با استفاده از مدل استخراج‌شده، پیش‌بینی پیوند را انجام دهند. با فرض داشتن یک شبکهٔ هدف مثل $G=(V,E)$، این روش‌ها یک تابع هدف ساخته شده را بهینه می‌کنند تا یک مدل از مجموعهٔ پارامترهای $\Theta$ بسازند که بتواند به بهترین شکل بر شبکهٔ هدف مطابق شود. سپس احتمال وجود و یا تشکیل یک پیوند ناموجود مثل $(i,j)$ با احتمال شرطی
$P(A_{ij}=1|\Theta)$
تخمین زده می‌شود. این روش‌ها به سه دستهٔ اصلی مدل احتمالاتی رابطه‌ای\LTRfootnote{\lr{Probabilistic Relational} Model} یا $PRM$، مدل احتمالاتی موجودیت-رابطه‌ای\LTRfootnote{\lr{Probabilistic Entity Relationship Model}} یا $PERM$، و مدل تصادفی رابطه‌ای\LTRfootnote{\lr{Stochastic Relational Model}} یا $SRM$ تقسیم‌بندی می‌شوند \cite{friedman1999learning}.

\subsection{معیارهای مبتنی بر نظریه اجتماعی}
شاخص‌هایی که پیش از این بیان شد، صرفا از گره و همبندی استفاده می‌کردند. شاخص‌های پیش‌بینی پیوند که بر پایهٔ نظریهٔ اجتماعی استوار هستند، می‌توانند عملکرد را با گرفتن اطلاعات تعاملات اجتماعی به ویژه در شبکه‌های با مقیاس بزرگ بهبود بخشند\cite{song2009scalable}. این اطلاعات می‌توانند مفاهیمی همچون روابط قوی\LTRfootnote{Strong ties} ، روابط ضعیف\LTRfootnote{Weak ties} ، انجمن‌ها و تعادل ساختاری باشد.
والورد-ریبازا\LTRfootnote{Valverde-Rebaza} و لوپز\LTRfootnote{Lopes}، همبندی و اطلاعات انجمن را با در نظر گرفتن علاقه و رفتارهای کابران ترکیب کردند و در نهایت پیوندهای آینده در توییتر را پیش‌بینی کردند\cite{valverde2013exploiting}. این نشان می‌دهد که این روش می‌تواند به شکلی کارآمد در بهبود عملکرد پیش‌بینی پیوند در شبکه‌های اجتماعی در مقیاس بزرگ موثر واقع شود. 
لیو\LTRfootnote{Liu} و همکاران یک مدل پیش‌بینی پیوند بر اساس روابط ضعیف و مرکزیت گره‌ها\LTRfootnote{Node Centeralities} ارائه کردند. مرکزیت، مبین اهمیت و تاثیرگذاری بیشتر در شبکه است و همچنین برای بهبود دقت پیش‌بینی از مفهوم رابطهٔ ضعیف استفاده شده‌است. بنابراین، هر یک از همسایگان مشترک گره‌ها متناسب با مرکزیت خود بر روی پیش‌بینی پیوند اثر خواهند داشت. این مدل به صورت زیر تعریف شده است:
\begin{equation}
LCW(x,y) = \sum _{z} (\omega (z).f(z))^{\beta}, \\\ f(z)=  \left\{
              \begin{array}{lcl}
                1 & \text{if} &  z \in \varGamma (x) \cap \varGamma (y)  ,\\
                0 & \text{if} & otherwise.
              \end{array} \right.
\end{equation}
که $\omega (z)$ به میزان مرکزیت هر گره در گراف شبکه اشاره می‌کند.

\subsection{روش‌های مبتنی بر یادگیری}
\begin{description}
  \item[روش‌های طبقه‌بندی مبتنی بر ویژگی]\LTRfootnote{Feature-based Classification}\textbf{:}
در گراف $G(V, E)$ که $x$ و $y$ گره‌های آن بوده ($ x,y \in V$) و $l^{(x,y)}$ برچسب زوج گره‌های نمونه $(x,y)$ هستند. در پیش‌بینی پیوند، هر جفت غیرمتصل گره‌ها به یک نمونه شامل برچسب کلاس و ویژگی‌های توصیف زوج گره‌ها تطبیق می‌یابد. بنابراین، یک جفت از گره‌ها می تواند به عنوان مثبت برچسب شود اگر یک پیوند اتصال گره‌ها وجود داشته باشد، در غیر این صورت، به عنوان منفی برچسب می‌شود. برچسب $x$ و $y$ به شرح زیر است:
\begin{equation}
            l^{(x,y)}=  \left\{
              \begin{array}{lcl}
                +1 & \text{if} &  (x,y) \in E  ,\\
                -1 & \text{if} &  (x,y) \notin E
              \end{array} \right.
\end{equation}
در این حالت پیش‌بینی پیوند شبیه طبقه‌بندی دودویی است و می‌توان از الگوریتم‌های طبقه‌بندی مانند درخت تصمیم\LTRfootnote{Decision Tree}، شبکه عصبی\LTRfootnote{Neaural Network}، ماشین بردار پشتیبان\LTRfootnote{Support Vector Machine} و غیره استفاده کرد.
  \item[روش‌های گراف احتمالاتی]:\ 
در یک شبکه اجتماعی، به پیوند میان هر زوج گره می‌توان مقدار احتمالی را نسبت داد مانند شباهت همبندی یا احتمال انتقال در ولگشت. این روش‌ها یک مدل احتمالاتی ایجاد می‌کنند و با کمک یال‌هایی که قبلا مشاهده شده‌اند، پارامتر لازم را تنظیم کرده و به پیش‌بینی پیوند می‌پردازند. این مدل‌ها از ویژگی‌های گره‌ها و یال‌ها به منظور مدل‌سازی توزیع احتمال توأم موجودیت‌ها و یال‌های میان آن‌ها استفاده می‌کنند. 
  \item[روش‌های تجزیه ماتریس]:\ 
منون\LTRfootnote{Menon} و همکاران پیش‌بینی پیوند را به مانند یک مساله تکمیل ماتریس معرفی کردند و روش تجزیه به عامل‌های ماتریس را به منظور پیش‌بینی پیوند گسترش دادند\cite{menon2011link}.

\end{description}

