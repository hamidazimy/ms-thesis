\newpage
\newcommand*{\framework}{چهارچوب }
\newcommand*{\proximity}{قرابت }
\newcommand*{\likelihood}{شانس }
\newcommand*{\issue}{موضوع }
\newcommand*{\regular}{باقاعده }
\newcommand*{\quasilocal}{شبه محلی }
\newcommand*{\topology}{هم‌بندی }
\newcommand*{\collaborationnetworks}{شبکه‌های همکاری }
\newcommand*{\cosinesimilarity}{تشابه کسینوسی }
\newcommand*{\HubPromoted}{حداقلی }
\newcommand*{\hubs}{هاب‌ها }
\newcommand*{\HubDepressed}{حداکثری }
\newcommand*{\scalefree}{مستقل از اندازه }
\chapter{بخش دوم}
\section{مقدمه}
\textbf{الگوریتم‌های برپایه‌ٔ شباهت}

 ساده‌ترین \framework متدهای پیش‌بینی پیوند، الگوریتم‌های بر پایه‌ٔ شباهت هستند به طوری که هر جفت از گره‌های $x$ و $y$، به امتیازی از $S_{xy}$ تخصیص داده‌ شده‌است که به طور مستقیم شباهت بین $x$ و $y$ معنا شده‌است (یا در ادبیات \proximity نامیده می‌شود).
 تمام پیوندهای دیده نشده با توجه به امتیازهای خود رتبه‌بندی می‌شوند و پیوند‌هایی که با گره‌های مشابه بیشتری در ارتباط 
هستند \likelihood وجود بالاتری برایشان تصور می‌شود.
با وجود سادگی آن، مطالعه بر روی الگوریتم‌های برپایه‌ی شباهت خود \issue مهمی است. در حقیقت، تعریف شباهت گره یک چالش کوچک اما با اهمیت است. شاخص تشابه می‌تواند بسیار آسان یا بسیار پیچیده باشد و برای برخی شبکه‌ها به خوبی پاسخگو می‌باشد و این در حالی است که برای بعضی شبکه‌های دیگر با شکست مواجه می‌شود.
علاوه بر این، شباهت‌ها می‌توانند در راه‌های پیشرفته‌تری مانند به صورت محلی تحت فیلترهای مشترک** \framework یکپارچه به کار گرفته شوند.
پ.ن** 
فیلترهای مشترک، فرآنیدی از فیلتر برای اطلاعات یا الگوها با استفاده از روش‌هایی شامل اشتراک بین چندین عامل نظیر نظرات، منابع داده‌ها و غیره  است.

شباهت گره می‌تواند با استفاده از ویژگی‌های اساسی گره‌ها تعریف شود. دو گره مشابه در نظر گرفته می‌شوند اگر که ویژگی‌های مشترک زیادی با یکدیگر داشته باشند. ازآنجایی‌که ویژگی‌های گره‌ها به طور کلی مخفی هستند در نتیجه ما بر گروه دیگری از شاخص شباهت با نام شباهت ساختاری** متمرکز میشویم که فقط بر پایه‌ی ساختار شبکه استوار است.
شاخص‌های شباهت ساختاری را می‌توان به روش‌های متعددی طبقه‌بندی کرد، از جمله طبقه‌بندی‌ها محلی در مقابل سراسری**،بدون پارامتر در قیاس با وابسته به پارامتر، وابسته به گره در مقابل وابسته به مسیر و غیره است.
   شاخص‌های شباهت همچنین می‌توانند از منظری دیگر به هم‌ارزی ساختاری و هم‌ارزی \regular طبقه‌بندی شوند.
 روش پیشین یک فرض نهان  را متصور شده‌است؟؟؟ که لینک خودش شباهت بین دو نقطهٔ نهایی را نشان داد (به عنوان نمونه به شاخص لنچ-هولم- نیوتون (۳۲) و انتقال شباهت (۳۳) نگاه کنید)؟؟؟ در حالی که در دومی فرض می‌شود که دو گره مشابه هستند اگر همسایه‌های آن‌ها مشابه باشند.

به خوانندگان توصیه می‌شود که مرجع شماره ۳۴ را برای تعریف ریاضی هم‌ارزی باقاعده و مرجع ۳۵ را برای کاربرد جدید پیش‌بینی توابع پروتیین مورد مطالعه قرار گیرد.؟؟؟

در این‌جا ساده‌ترین روش انتخاب شده‌است،  که در آن ۲۰ شاخص شباهت به ۳ دسته طبقه‌بندی می‌شوند که شامل ۱۰ شاخص محلی و ۷ شاخص سراسری و نیز ۳ شاخص \quasilocal است که به اطلاعات \topology سراسری نیاز ندارد اما اطلاعات بیشتری  را نسبت به شاخص‌های محلی  به کار میبرد.؟؟؟

\textbf{۳.۱: شاخص شباهت محلی}
(۱) همسایگان مشترک** ($CN$)

 برای یک گره با نام $x$،مجموعه‌ای از همسایه‌های $x$ با $\Gamma{x}$ نشان داده می‌شود.
دو گره $x$ و $y$ چنانچه همسایه‌های مشترک زیادی با یکدیگر داشته باشند با احتمال خوبی به یکدیگر نیز مرتبط هستند، ساده‌ترین اندازه‌گیری این همپوشانی همسایه‌ای شمار؟؟؟ مستقیم است، یعنی ؟؟؟
\begin{equation}
S_{xy}^{CN}=|\Gamma{x}\cap\Gamma{y}|
\end{equation}

که|$Q$| تعداد اعضا مجموعهٔ $Q$ است. ؟؟؟بدیهی است 
$S_{xy}=(A^2)_{xy}$،
که در آن $A$ ماتریس مجاورت است و چنانچه $x$ و $y$ به طور مستقیم با یکدیگر در ارتباط باشند $A_{xy}=1$ و در غیر اینصورت $A_{xy}=0$
توجه شود که همچنین $(A^2)_{xy}$ بیانگر تعداد مسیرهای مختلف با طول ۲ است که $x$ . $y$ را به هم متصل میکنند.
نیومن (۳۶)؟؟؟ این کمیت را در مطالعات خود در زمینهٔ \collaborationnetworks مورد بررسی قرار داد، نشان داد همبستگی مثبتی بین تعداد همسایگان مشترک و احتمال که دو محقق د ر آینده همکاری خواهند داشت.؟؟؟کسینتز و واتس (۱۴) مقیاس بزرگی شبکه‌های اجتماعی را تجزیه و تحلیل کردند که نشان می‌دهد دو دانشجو که دارای دوستان مشترک بسیاری هستند، دوست شدن آن‌ها در آینده از احتمال خوبی برخوردار است.
(۲) شاخص سالتون، به شکل زیر تعریف شده‌است:
\begin{equation}
S^{Salton}_{xy}=\frac{|\Gamma(x)\cap\Gamma(y)|}{\sqrt{k_{x}\times k_{y}}}
\end{equation}

در این‌جا $k_{x}$ نشان‌دهنده‌ی درجهٔ گره $x$ است. همچنین شاخص سالتون در ادبیات \cosinesimilarity نامیده می‌شود.

(۳) شاخص جاکارد، که این شاخص توسط جاکارد بالغ بر ۱۰۰ سال پیش ارائه شد به شکل زیر تعریف می‌شود:
\begin{equation}
S^{Jaccard}_{xy}=\frac{|\Gamma(x)\cap\Gamma(y)|}{|\Gamma(x)\cup\Gamma(y)|}
\end{equation}

(۴) شاخص سورنسن؟؟؟ که این شاخص به طور عمده برای داده جامعه محیط زیست مورد استفاده قرار می‌گرفته‌است؟؟؟ و به صورت زیر تعریف می‌شود:

\begin{equation}
S_{xy}^{S\phi rensen}=\frac{2|\Gamma(x)\cap\Gamma(y)|}{k_{x}+k_{y}}
\end{equation}

(۵) شاخص \HubPromoted ($HPI$) *** این شاخص برای تعیین همپوشانی ساختاری از جفت‌ لایه‌‌ها در شبکه‌های زیستی پیشنهاد شده است که به صورت زیر تعریف می‌شود: 

\begin{equation}
S_{xy}^{HPI}=\frac{|\Gamma(x)\cap\Gamma(y)|}{\min\{ k_{x},k_{y} \}}
\end{equation}

بر اساس این اندازه‌گیری، لینک‌های مجاور به \hubs محتمل‌تر هستند که نمرات بالاتری به آن‌ها تخصیص یابد به این دلیل که درجهٔ کوچک‌تر مخرج را تعیین می‌کند.

(۶) شاخص \HubDepressed ($HDI$) *** مشابه شاخص پیشین، که با درنظر گرفتن اندازه‌گیری با تاثیر معکوس بر \hubs که به این شکل تعریف می‌شود:

\begin{equation}
S_{xy}^{HDI}=\frac{|\Gamma(x)\cap\Gamma(y)|}{\max\{ k_{x},k_{y} \}}
\end{equation}
 (۷) شاخص لایت-هولم-نیومن *** ($LHN1$) این شاخص، شباهت بیشتر را به زوج گره‌هایی که همسایگان بیشتری (نه در قیاس با بیشینه احتمال بلکه) در مقایسه با تعداد همسایگان موردانتظار دارند، تخصیص داده‌است که به صورت زیر تعریف می‌شود:
 \begin{equation}
S_{xy}^{LHN1}=\frac{|\Gamma(x)\cap\Gamma(y)|}{k_{x}\times k_{y}}
\end{equation}
همان‌طور که مشاهده شد مخرج کسر بالا ($k_{x} \times k_{y}$) متناسب با تعداد مورد انتظار همسایگان مشترک گره‌های $x$ و $y$ در مدل پیکربندی است. شاخص مذکور به اختصار $LHN1$ نام‌گذاری شده‌است تا از شاخص دیگری که لایت-هولم-نیومن پیشنهاد داده‌اند ($LHN2$) تمییز داده شود.
(۸) معیار وابستگی ترجیحی*** ($PA$). این معیار می‌تواند به منظور ایجاد تحول در شبکه‌های \scalefree به کار گرفته‌شود، که در آن احتمال اینکه یک پیوند جدید به گره $x$ متصل شود متناسب با $k_{x}$ است. راه‌کار مشابه همچنین می‌تواند منجر به شبکه‌های \scalefree بدون رشد شود. که در آن در هر مرحله، یک پیوند قدیمی حذف شده‌ و یک پیوند جدید تولید می‌شود و احتمال آنکه پیوند جدید دو گره $x$ و $y$ را به یکدیگر متصل کند متناسب با $k_{x} \times k_{y}$ است.
\begin{equation}
S_{xy}^{PA}=k_{x}\times k_{y}
\end{equation}
که به طور گسترده‌ای برای تعیین کمیت اهمیت کاربردی موضوع پیوندها به پویایی‌های مبتنی بر شبکه مورد استفاده قرار می‌گرفته‌است، مانند نفوذ، همگام‌سازی و انتقال. توجه داشته باشید که این معیار به اطلاعات همسایگان هر گره نیازی ندارد، در نتیجه از حداقل پیچیدگی محاسباتی برخوردار است.

(۹) معیار آدامیک-آدار. این معیار با شمارش سادهٔ همسایه‌های مشترک با اختصاص وزن بیشتر به همسایگانی که خود آن‌ها دارای همسایگان کمتری هستند، تعریف می‌شود. 
\begin{equation}
S_{xy}^{AA}= \sum_{z\in\Gamma(x)\cap\Gamma(y)}{\frac{1}{\log k_{z}}}.
\end{equation}
(۱۰) معیار تخصیص منابع ($RA$). این معیار توسط منابع پویای تخصیص در شبکه‌های پیچیده پرورش یافت. یک جفت گره را در نظر بگیرید، گره $x$ و $y$، که به صورت مستقیم به یکدیگر مرتبط نیستند. گره $x$ می‌تواند تعدادی منبع را به گره $y$ به وسیلهٔ همسایه‌های مشترک آن دو که نقش فرستنده را ایفا می‌کنند، ارسال کند. در ساده‌ترین حالت، فرض می‌کنیم که هر فرستنده دارای یک واحد از منابع است و به یک اندازه آن را به تمام همسایگان خود توزیع می‌کند. شباهت بین $x$ و $y$ می‌تواند به عنوان مقدار منابعی که $y$ از $x$ دریافت کرده‌است تعریف شود:
\begin{equation}
S_{xy}^{RA}= \sum_{z\in\Gamma(x)\cap\Gamma(y)}{\frac{1}{k_{z}}}.
\end{equation}

واضح است که این اندازه‌گیری متقارن (به عنوان نمونه $S_{yx} = S_{xy}$) است. توجه داشته باشید که، فارغ از انگیزه‌های متفاوت هر دو معیار $AA$ و $RA$ فرم شبیه به هم دارند.؟؟؟ در واقع، هر دو معیار سهم تاثیر همسایه‌های مشترک با درجهٔ بالا را کاهش می‌دهند. معیار $AA$ فرمی اینگونه دارد $(\log k_{z})^{-1}$ در حالی که معیار $RA$ فرم $k_{z}^{-1}$ را دارد. فرم؟؟؟؟ ;) :*
