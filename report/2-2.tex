\section{تعاریف، اصول و مبانی نظری}
یک گراف ساده بدون جهت $G(V, E)$ را در نظر بگیرید، که در آن $V$ مجموعهٔ گره‌ها و $E$ مجموعهٔ یال‌های بین گره‌هاست. طبق تعریف گراف ساده، یال‌های چندگانه\LTRfootnote{Multiple Edge} و حلقه‌ها\LTRfootnote{Self Loops} در این گراف مجاز نیستند. مجموعهٔ تمام یال‌های ممکن بین تمام گره‌ها را با $U$ نشان می‌دهیم که تعداد این یال‌ها
$\frac{|V|\times(|V|-1)}{2}$
است. نماد $|V|$ به معنی تعداد اعضای مجموعهٔ $V$ است که در واقع تعداد یال‌ها را نشان می‌دهد. در نتیجه مجموعهٔ یال‌های ناموجود در گراف $G$ برابر است با $U - E$. ما فرض می‌کنیم که در این مجموعه تعدادی پیوند گم‌شده وجود دارند (پیوندهایی که دیده نشده‌اند و یا ممکن است در آینده به وجود بیایند) و هدف ما این است که این پیوندهای گم‌شده را پیدا کنیم و پیش‌بینی کنیم.

در حالت کلی ما نمی‌دانیم کدام‌یک از پیوندها دیده نشده‌اند یا ممکن است در آینده به وجود بیایند، چون در غیر این صورت دیگر نیازی به پیش‌بینی نداشتیم. بنابراین برای آزمودن دقت الگوریتم‌ها، مجموعه‌ٔ پیوندهای گراف $G$ یعنی $E$ را به صورت تصادفی به دو بخش افراز می‌کنیم: بخش اول دادهٔ آموزش که با $E'$ نمایش می‌دهیم و به عنوان دادهٔ شناخته از آن استفاده می‌کنیم؛ و بخش دوم که با $E''$ نمایش داده می‌شود و داده‌ٔ آزمون ماست و برای آزمودن دقت الگوریتم‌ها استفاده می‌شود و هیچ اطلاعاتی از آن نمی‌تواند در فرآیند پیش‌بینی استفاده شود. طبق تعریف افراز واضح است که $E' \cup E'' = E$ و $E' \cap E'' = E$ برقرارند. مزیت این افراز تصادفی این است که نسبت تقسیم، به تعداد تکرارها وابسته نیست. اما با این روش، بعضی پیوندها ممکن است هیچ‌گاه در مجموعهٔ آزمون قرار نگیرند، و از طرف دیگر بعضی بیش از یک بار در مجموعهٔ آزمون قرار بگیرند که همین باعث سوگیری\LTRfootnote{Bias} آماری می‌شود. برای بر طرف کردن این محدودیت می‌توان از روش \textit{ارزیابی متقاطع $K$-قسمتی}\LTRfootnote{K-fold cross-validation} استفاده کرد که در آن، مجموعه‌ٔ پیوند‌های مشاهده شده ($E$) به صورت تصادفی به $K$ زیرمجموعه افراز می‌شوند. هر بار یکی از این زیرمجموعه‌ها به عنوان مجموعهٔ آزمون انتخاب می‌شود و اجتماع بقیهٔ $K - 1$ زیرمجموعه به عنوان مجموعهٔ آزمون استفاده می‌شود. این فرآیند $K$ بار تکرار می‌شود و بنابراین هر زیرمجموعه دقیقاً یک بار به عنوان مجموعهٔ آزمون انتخاب شود. با این کار، تمام پیوندها هم برای آموزش و هم برای اعتبارسنجی مورد استفاده قرار می‌گیرند و هر پیوند دقیقاً یک بار برای پیش‌بینی به کار می‌رود. به وضوح هر چقدر $K$ بیشتر باشد، سوگیری آماری کمتری خواهیم داشت، اما از طرف دیگر هزینهٔ محاسباتی بیشتری را می‌بایست متقبل شویم. بعضی شواهد تجربی پیشنهاد می‌کنند که استفاده از روش ارزیابی متقاطع ۱۰-قسمتی، توازن\LTRfootnote{Trade-off} بسیار خوبی بین هزینه و کارایی ایجاد می‌کند\cite{breiman1992submodel} و \cite{kohavi1995study}. در این پژوهش نیز از همین روش استفاده خواهد شد.
 %یک حالت نهایی برای این روش، موسوم به \lr{\textit{leave-one-out}} است که هر بار یک پیوند را به عنوان مجموعهٔ آزمون در نظر می‌گیرد و نام دیگر آن نیز \lr{$|V|$\textit{-fold cross-validation}} است.

همان‌طور که پیش‌تر گفته شد، هدف ما بررسی تاثیر مشارکت وزن پیوندها در کیفیت پیش‌بینی پیوند است، بنابراین گراف ورودی ما یک گراف وزن‌دار خواهد بود. گسترش تعریف این مسئله به مسئله پیش‌بینی پیوند وزن‌دار بسیار ساده است. تنها فرض اضافه شده این است که به هر پیوند یک عدد مثبت به عنوان وزن آن پیوند اضافه شده‌است. در ماتریس مجاورت گراف‌های بدون وزن، اعداد ۱ و ۰ به ترتیب به معنی وجود و عدم وجود پیوند هستند، در حالی که در ماتریس مجاورت گراف‌های وزن‌دار، هر عدد وزن پیوند متناظر را مشخص می‌کند.





