\section{اهداف و دستاوردهای پژوهش}

همان‌طور که در بخش قبل خاطر نشان شد، بهبود کارایی روش‌های پیش‌بینی پیوند موضوع بسیار مهمی است که توجه ویژه‌ای را طلب می‌کند. برای افزایش دقت این الگوریتم‌ها، می‌بایست تا جای ممکن اطلاعاتی را که یک شبکه می‌تواند در اختیار ما قرار دهد، استخراج، و از آن‌ها به نحو احسن استفاده کرد.

%یکی از مواردی که می‌تواند ما را در رسیدن به کارایی‌های بالاتر یاری کند، در نظر گرفتن وزن پیوندهاست. در بسیاری از شبکه‌ها، پیوندهای بین گره‌ها دارای وزن هستند که این وزن، قدرت پیوند و اهمیت پیوند رابطه را مشخص می‌کند.  این پژوهش نیز می‌کوشد تا روش‌های پیش‌بینی پیوند را در زمینهٔ شبکه‌های وزن‌دار بررسی کند.

یکی از اطلاعاتی که معمولاً در شبکه‌ها به آن دسترسی داریم، اطلاعات وزن پیوندهاست. یعنی رابطهٔ بین گره‌ها در این شبکه‌ها فراتر از یک رابطهٔ دودویی\LTRfootnote{Binary Relation} است که فقط وجود یا عدم وجود پیوند را نشان دهد، به این معنی که به ما قدرت رابطه بین دو گره را نیز نشان می‌دهد. برای مثال در یک شبکهٔ اجتماعی، میزان تعامل بین دو کاربر می‌تواند وزن رابطهٔ آن‌ها باشد. یا در شبکه‌ٔ فرودگاه‌های کشور، تعداد پرواز بین دو فرودگاه را می‌توان به عنوان وزن پیوند ارتباطی بین آن‌ها در نظر گرفته شود. تلاش‌هایی در زمینهٔ استفاده از وزن پیوندها در پیش‌بینی انجام شده است که در بخش‌های بعدی به آن‌ها اشاره خواهد شد. یکی از اهداف این پژوهش بررسی این نکته است که استفاده از این وزن پیوندها چگونه می‌تواند به ما در بهبود کارایی روش‌ها کمک کند که در انتها به عنوان یکی از دستاوردهای این پژوهش دربارهٔ آن بحث خواهد شد.

یکی دیگر از اطلاعاتی که یک شبکه در اختیار ما قرار می‌دهد، اطلاعات ساختاری آن است و یکی از این اطلاعات ساختاری که می‌تواند از یک شبکه استخراج شود، اطلاعات انجمن‌های\LTRfootnote{Community} آن شبکه است. یکی دیگر از دست‌آوردهای پژوهش حاضر، این است که به وسیلهٔ ترکیب اطلاعات انجمن‌ها با اطلاعات وزن پیوندها، روش جدیدی ارائه می‌دهد که می‌تواند در مواردی که دربارهٔ آن‌ها به تفصیل بحث خواهد شد، کارایی روش‌های پیش‌بینی پیوند را بهبود بخشد.
