\newpage
\chapter{جمع‌بندی و پیشنهادها}
\newpage
\section{مقدمه}
با توجه به رشد روزافزون تولید بازی‌های رایانه‌ای و شبیه‌تر شدن این بازی‌ها به دنیای واقعی، نیاز به رفتارهای گروهی و هماهنگ که بین انسان متداول است احساس می‌شود. در این فصل ابتدا یک جمع‌بندی کلی از پژوهش انجام شده ارائه می‌دهیم و سپس پیشنهاداتی برای کارهای آینده مطرح می‌گردد.
\section{جمع‌بندی}
در این پژوهش، ما مفاهیم کلیدی مختلفی را که در شکل‌گیری رفتارهای هوشمند بین اعضای یک تیم یا گروه، نقش
عمده ایفا می‌کنند، بررسی کردیم. با در نظر گرفتن این نکات و پیاده‌سازی آن‌ها با توجه به نیازها، خصوصیات و
محدودیت‌های بازی مورد نظر، بازی ما می‌تواند از یک هوش مصنوعی گروهی منسجم و خوب بهره ببرد که موجب برتری آن
نسبت به بازی‌های دیگر خواهد شد و می‌تواند به بازیکن، حس طبیعی‌تر بودن بازی را القا کند و تجربه‌ی خوبی را برای
او رقم بزند.

در این پژوهش، ما ابتدا به سراغ بررسی انواع مختلف آرایش‌های نظامی، ویژگی‌های آن‌ها،
نحوه‌ی تخصیص نیروهای مختلف به جایگاه‌های آرایش، نحوه‌ی حرکت به سمت آرایش‌ها، صف‌آرایی، عبور از موانع مختلف و…
رفتیم و آن‌ها را مورد بررسی قرار دادیم. در ادامه مفهومی به اسم راه‌نشان را معرفی کردیم و از آن برای تعیین
نحوه‌ی حرکت نیروها و یافتن نقاط امن و نقاط کمین و… استفاده نمودیم. در بخش بعدی به سراغ مفهومی با اسم
نقشه‌ی تأثیر رفتیم و نقش آن را در تشخیص قسمت‌های مختلف نیروهای دشمن و نقاط آسیب‌پذیر آن بررسی کردیم و
راهکارهایی برای حمله و درگیری مؤثر با دشمن معرفی کردیم.

در بخش‌های بعد نیز به بررسی تاکتیک‌های تیمی و همکاری اعضا پرداختیم و تاکتیک‌های متمرکز و غیرمتمرکز را بررسی
کردیم. گفتیم که تاکتیک‌های تیمی غیرمتمرکز بسط ساده‌ای از هوش انفرادی خواهد بود و هر عامل با توجه به وضعیت خود
و وضعیت هم‌تیمی‌های خود، اقدام به انجام بهترین عمل خواهد کرد. اما در تاکتیک‌های متمرکز، یک عامل
به عنوان فرمانده، به بقیه‌ی اعضای گروه فرمان خواهد داد. در ادامه بررسی کردیم که انواع مختلف فرماندهی
چه خصوصیاتی دارند و مانورهای مختلف چگونه انجام می‌گیرند.

در فصل بعد نیز به توضیح مختصری درباره‌ی پیاده‌سازی انجام گرفته در این زمینه‌ها پرداختیم.


\section{پیشنهادات}
در ادامه، پیشنهاداتی برای کارهای آینده به جهت بهبود موارد بررسی شده، ارائه می‌گردد.
\subsection{روش سلسله‌مراتبی}
ما در این پژوهش دو روش متمرکز و غیر متمرکز را برای فرماندهی بررسی کردیم. برای کارهای آینده می‌توان  بر روی روش‌های ترکیبی کنترل و فرماندهی گروه تمرکز کرد. به عنوان مثال می‌توان از روش سلسله‌مراتبی استفاده کرد. منظور از روش سلسله‌مراتبی این است که هر یک اعضای گروه برای خود درجه‌ای دارند و در طول انجام عملیات‌ها فقط از عضو با درجه‌ی بیشتر دستور می‌گیرند و فقط به او پاسخگو هستند.
\subsection{استفاده از قوانین فازی}
ما در قسمت دوم فصل دو از رشته‌های بیتی برای تعیین امنیت نقاط مختلف نقشه استفاده کردیم و نقاط را به دو دسته‌ی امن و خطرناک تقسیم نمودیم. می‌توان برای تعیین امنیت نقاط از قوانین فازی استفاده کرد یعنی درجه‌ای از امنیت تعریف کرد و هر چه این مقدار بیشتر نقطه‌ی مورد نظر امن‌تر تلقی شود.
\newpage
