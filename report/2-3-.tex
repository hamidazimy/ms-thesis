\subsection{روش‌های \MaximumLikelihood}
یک دسته از روش‌های پیش‌بینی پیوند، دسته روش‌های مبتنی بر تخمین \MaximumLikelihood هستند. این روش‌ها ابتدا یک سری اصول ساختاری برای ساختار شبکه در نظر می‌گیرند و با بیشینه کردن شباهت ساختار دیده شده از شبکه، مجموعه قوانین و پارامترهای مشخصی را به دست می‌آورند. سپس، احتمال وجود هر کدام از پیوندهای دیده‌نشده می‌تواند با توجه به این قوانین و پارامترها محاسبه شود.

از نقطه نظر کاربردهای تجربی، یک مشکل اساسی برای روش‌های \MaximumLikelihood این است که بسیار زمان‌بر هستند. یک الگوریتم خوب طراحی شده از این نوع، قادر است با شبکه‌هایی با حداکثر چند هزار گره در زمان قابل قبول کار کند، اما وقتی با گراف‌های بسیار بزرگ شبکه‌های اجتماعی برخط که معمولاً از میلیون‌ها گره تشکیل شده‌اند مواجه می‌شود، قادر به انجام عملیات نخواهد بود. علاوه بر این، روش‌های \MaximumLikelihood معمولاً جزو دقیق‌ترین و بهترین روش‌های پیش‌بینی پیوند نیستند. با این حال، این روش‌ها دید ارزشمندی از ساختار شبکه برای ما به ارمغان می‌آورند که در روش‌های دیگر مثل روش‌های بر پایهٔ شباهت یا روش‌های احتمالاتی نخواهیم داشت.

\subsection{مدل‌های احتمالاتی}
هدف روش‌های احتمالاتی این است که ساختار زیرین یک شبکه را استخراج کند و سپس با استفاده از مدل استخراج‌شده، پیش‌بینی پیوند را انجام دهند. با فرض داشتن یک شبکهٔ هدف مثل $G=(V,E)$، این روش‌ها یک تابع هدف ساخته شده را بهینه می‌کنند تا یک مدل از مجموعهٔ پارامترهای $\Theta$ بسازند که بتواند به بهترین شکل بر شبکهٔ هدف مطابق شود. سپس احتمال وجود و یا تشکیل یک پیوند ناموجود مثل $(i,j)$ با احتمال شرطی
$P(A_{ij}=1|\Theta)$
تخمین زده می‌شود. این روش‌ها به سه دستهٔ اصلی مدل احتمالاتی رابطه‌ای\LTRfootnote{\lr{Probabilistic Relational} Model} یا $PRM$، مدل احتمالاتی موجودیت-رابطه‌ای\LTRfootnote{\lr{Probabilistic Entity Relationship Model}} یا $PERM$، و مدل تصادفی رابطه‌ای\LTRfootnote{\lr{Stochastic Relational Model}} یا $SRM$ تقسیم‌بندی می‌شوند \cite{friedman1999learning}.

\subsection{معیارهای مبتنی بر نظریه اجتماعی}
شاخص‌هایی که پیش از این بیان شد، صرفا از گره و همبندی استفاده می‌کردند. شاخص‌های پیش‌بینی پیوند که بر پایهٔ نظریهٔ اجتماعی استوار هستند، می‌توانند عملکرد را با گرفتن اطلاعات تعاملات اجتماعی به ویژه در شبکه‌های با مقیاس بزرگ بهبود بخشند\cite{song2009scalable}. این اطلاعات می‌توانند مفاهیمی همچون روابط قوی\LTRfootnote{Strong ties} ، روابط ضعیف\LTRfootnote{Weak ties} ، انجمن‌ها و تعادل ساختاری باشد.
والورد-ریبازا\LTRfootnote{Valverde-Rebaza} و لوپز\LTRfootnote{Lopes}، همبندی و اطلاعات انجمن را با در نظر گرفتن علاقه و رفتارهای کابران ترکیب کردند و در نهایت پیوندهای آینده در توییتر را پیش‌بینی کردند\cite{valverde2013exploiting}. این نشان می‌دهد که این روش می‌تواند به شکلی کارآمد در بهبود عملکرد پیش‌بینی پیوند در شبکه‌های اجتماعی در مقیاس بزرگ موثر واقع شود. 
لیو\LTRfootnote{Liu} و همکاران یک مدل پیش‌بینی پیوند بر اساس روابط ضعیف و مرکزیت گره‌ها\LTRfootnote{Node Centeralities} ارائه کردند. مرکزیت، مبین اهمیت و تاثیرگذاری بیشتر در شبکه است و همچنین برای بهبود دقت پیش‌بینی از مفهوم رابطهٔ ضعیف استفاده شده‌است. بنابراین، هر یک از همسایگان مشترک گره‌ها متناسب با مرکزیت خود بر روی پیش‌بینی پیوند اثر خواهند داشت. این مدل به صورت زیر تعریف شده است:
\begin{equation}
LCW(x,y) = \sum _{z} (\omega (z).f(z))^{\beta}, \\\ f(z)=  \left\{
              \begin{array}{lcl}
                1 & \text{if} &  z \in \varGamma (x) \cap \varGamma (y)  ,\\
                0 & \text{if} & otherwise.
              \end{array} \right.
\end{equation}
که $\omega (z)$ به میزان مرکزیت هر گره در گراف شبکه اشاره می‌کند.

\subsection{روش‌های مبتنی بر یادگیری}
\begin{description}
  \item[روش‌های طبقه‌بندی مبتنی بر ویژگی]\LTRfootnote{Feature-based Classification}\textbf{:}
در گراف $G(V, E)$ که $x$ و $y$ گره‌های آن بوده ($ x,y \in V$) و $l^{(x,y)}$ برچسب زوج گره‌های نمونه $(x,y)$ هستند. در پیش‌بینی پیوند، هر جفت غیرمتصل گره‌ها به یک نمونه شامل برچسب کلاس و ویژگی‌های توصیف زوج گره‌ها تطبیق می‌یابد. بنابراین، یک جفت از گره‌ها می تواند به عنوان مثبت برچسب شود اگر یک پیوند اتصال گره‌ها وجود داشته باشد، در غیر این صورت، به عنوان منفی برچسب می‌شود. برچسب $x$ و $y$ به شرح زیر است:
\begin{equation}
            l^{(x,y)}=  \left\{
              \begin{array}{lcl}
                +1 & \text{if} &  (x,y) \in E  ,\\
                -1 & \text{if} &  (x,y) \notin E
              \end{array} \right.
\end{equation}
در این حالت پیش‌بینی پیوند شبیه طبقه‌بندی دودویی است و می‌توان از الگوریتم‌های طبقه‌بندی مانند درخت تصمیم\LTRfootnote{Decision Tree}، شبکه عصبی\LTRfootnote{Neaural Network}، ماشین بردار پشتیبان\LTRfootnote{Support Vector Machine} و غیره استفاده کرد.
  \item[روش‌های گراف احتمالاتی]:\ 
در یک شبکه اجتماعی، به پیوند میان هر زوج گره می‌توان مقدار احتمالی را نسبت داد مانند شباهت همبندی یا احتمال انتقال در ولگشت. این روش‌ها یک مدل احتمالاتی ایجاد می‌کنند و با کمک یال‌هایی که قبلا مشاهده شده‌اند، پارامتر لازم را تنظیم کرده و به پیش‌بینی پیوند می‌پردازند. این مدل‌ها از ویژگی‌های گره‌ها و یال‌ها به منظور مدل‌سازی توزیع احتمال توأم موجودیت‌ها و یال‌های میان آن‌ها استفاده می‌کنند. 
  \item[روش‌های تجزیه ماتریس]:\ 
منون\LTRfootnote{Menon} و همکاران پیش‌بینی پیوند را به مانند یک مساله تکمیل ماتریس معرفی کردند و روش تجزیه به عامل‌های ماتریس را به منظور پیش‌بینی پیوند گسترش دادند\cite{menon2011link}.

\end{description}
