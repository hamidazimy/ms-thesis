\section{جمع‌بندی}

هدف اصلی از انجام این پژوهش، کمک به بهبود روش‌های پیش‌بینی پیوند موجود بوده است. همان‌طور که عنوان شد، پیش‌بینی پیوند یک مسئلهٔ اساسی و مهم در دانش اطلاعات مدرن است که در حوزه‌های مختلف از تحلیل شبکه‌های اجتماعی گرفته تا  زیست‌شناسی و غیره کاربرد دارد و همین موضوع، نیاز به روش‌های کارا برای حل این مسئله را روشن می‌کند. بر اساس همین هدف، تلاش‌هایی در همین راستا صورت گرفته است که لیست دستاوردهای آن به شرح زیر است:
\begin{itemize}
  \item مطالعهٔ کاملی از انواع روش‌های پیش‌بینی پیوند انجام شد که دسته‌بندی و توضیحات مربوط به هر یک از روش‌ها در فصل مروری بر ادبیات آورده شده‌است.
  \item ایدهٔ اصلی این پژوهش که استفاده از وزن پیوندها به همراه اطلاعات انجمن‌ها بود مطرح شد و به همین منظور پیشینه مختصری از روش‌ها تشخیص انجمن‌ها و روش مورد نظر این پژوهش ارائه شد.
  \item روش پیشنهادی در این پژوهش که معرفی شد. توضیح داده شد که چرا اطلاعات انجمن‌ها برای پیش‌بینی پیوند مفیدند. روش اصلی پیشنهادی محدود کردن روش‌های پیش‌بینی پیوند به پیش‌بینی در داخل انجمن‌ها بود. دلیل این امر این است که از طرفی معمولاً گره‌ها تمایل بیشتری به ایجاد پیوند با گره‌های هم‌انجمن خود دارند و با احتمال بیشتری به آن‌ها متصل خواهند شد و از طرف دیگر تعداد یال‌های بالقوه درون انجمن‌ها بسیار کمتر از یال‌های بالقوه خارج از انجمن‌هاست و همین دو نکته باعث می‌شود که پیش‌بینی داخل انجمن بتواند کمک خوبی به بهبود کارایی روش‌های پیش‌بینی پیوند بکند.
    \item روش پیشنهادی به دو گام محاسبهٔ شاخص شباهت و تشخیص انجمن تقسیم شد. توضیح داده شد که استفاده از وزن پیوندها در هر کدام از این گام‌ها امکان‌پذیر است. در نتیجه با توجه به استفاده کردن یا نکردن از وزن پیوندها در هر کدام از این دو گام، روش پیشنهادی به چهار روش تبدیل می‌شود که هر کدام از این چهار روش می‌توانند با توجه به ویژگی‌های شبکه‌ای که در آن قرار است استفاده شوند، کارایی‌های متفاوتی از خود نشان دهند.
    \item برای انجام آزمایش و آزمودن روش‌های پیشنهادی، یک نوع شبکهٔ مصنوعی پارامتری به نام شبکه‌های \lr{LFR} انتخاب شد و پیشینه این شبکه‌ها بررسی شد. شبکه‌های \lr{LFR} نوعی شبکهٔ مستقل از اندازه هستند که بررسی آن‌ها می‌تواند نکات ارزشمندی در اختیار ما بگذارد. از آن‌جایی که این شبکه‌ها، شبکه‌های پارامتری هستند، بررسی روش‌های پیشنهادی روی فضای پارامتری این شبکه‌ها که متشکل از دو پارامتر $\mu_t$ و $\mu_w$ است انجام شد. این پارامترها به ترتیب نحوهٔ توزیع یال‌ها در درون و بیرون از انجمن‌ها، و نحوهٔ توزیع وزن یال‌ها باز هم در درون و بیرون از انجمن‌ها را تعیین می‌کنند.
    \item در نهایت بعد از بررسی آزمایشات انجام شده، نتیجه‌گیری شد که در حالت‌هایی که $\mu_t$ از $\mu_w$ بیشتر است، یعنی وقتی که یال‌های درون انجمن‌ها قوی‌تر از یال‌های بیرون انجمن‌ها هستند، روش‌هایی که از تشخیص انجمن وزن‌دار استفاده می‌کنند کارایی بهتری از حالت معمولی (بدون استفاده از انجمن‌ها دارند) ولی روش‌هایی که از تشخیص انجمن بدون وزن استفاده می‌کنند کارایی مطلوبی ندارند به این دلیل که وزن زیاد یال‌ها درون انجمن‌ها، به تشخیص انجمن کمک می‌کند. از طرف دیگر در حالت‌هایی که $\mu_w$ از $\mu_t$ بیشتر است، روش‌هایی که از تشخیص انجمن بدون وزن استفاده می‌کنند بسیار بهتر از حالت معمولی و دو روش دیگر عمل می‌کنند به این دلیل که وزن زیاد یال‌های خارج از انجمن، باعت به خط افتادن تشخیص انجمن وزن‌دار می‌شود.    
\end{itemize}
