\section{معرفی مجموعه داده‌ها}
هدف از این پژوهش، بررسی کارایی روش‌های پیشنهادی با توجه به پارامترهای مختلف در یک نوع شبکهٔ مصنوعی\LTRfootnote{synthesis network} است. مجموعه داده‌های آزمایشی در این پژوهش، یک نوع شبکهٔ مصنوعی موسوم به شبکه‌های \lr{LFR} هستند. این شبکه‌ها در اصل به عنوان نوعی شبکهٔ محک\LTRfootnote{benchmark} برای روش‌های تشخیص انجمن‌ها توسط لانچیکینِتی\LTRfootnote{Lancichinetti}، فُرتوناتو \LTRfootnote{Fortunato} و رادیکی\LTRfootnote{Radicchi} در سال ۲۰۰۸ معرفی شدند \cite{lancichinetti2008benchmark}. شبکه‌های \lr{LFR} از خانوادهٔ شبکه‌های \ScaleFree هستند و در آن‌ها فرض می‌شود که توزیع‌های درجهٔ گره‌ها، \Strength گره‌ها و اندازه انجمن‌ها از توزیع \PowerLaw با پارامترهای $\gamma $ و $\beta $ تبعیت می‌کنند.
این شبکه‌ها، شبکه‌هایی مصنوعی هستند که با تعدادی پارامتر ورودی ساخته می‌شوند و انجمن‌های موجود در شبکه را نیز می‌دهد و می‌توان از آن برای مقایسه روش‌های تشخیص انجمن‌ها استفاده کرد. اما در این پژوهش استفاده‌ای که از این شبکه‌ها انجام خواهد گرفت، در مسئلهٔ پیش‌بینی پیوند است. به این صورت که تاثیر پارامترهای ورودی این شبکه‌ها روی دقت پیش‌بینی مورد بررسی قرار خواهد گرفت. ابتدا توضیحاتی در مورد این شبکه‌ها و پارامترهای ورودی آن‌ها داده خواهد شد.
همان‌طور که اشاره شد، این شبکه‌ها می‌توانند با هم مجموعه دلخواه از پارامترها ساخته شده و مورد استفاده قرار گیرند. پارامترهایی که می‌توان برای این شبکه‌ها در نظر گرفت پارامترهایی نظیر تعداد گره‌ها، میانگین درجه گره‌ها، بیشترین اندازهٔ هر انجمن و… هستند. در ادامه لیستی از این پارامترها آورده و توضیح داده خواهند شد:
\begin{itemize}
  \item $N$: اولین پارامتر برای تولید این شبکه‌ها، تعداد گره‌ها یا $N$ است.
  \item $k$: پارامتر مهم بعدی $k$ یا میانگین درجات گره‌های گراف است.
  \item $maxk$: پارامتر $maxk$ حداکثر درجهٔ گره‌های گراف را مشخص می‌کند.
  \item $\mu_t$:
  پارامتر بسیار مهم بعدی یعنی $\mu_t$ که به آن \Mut\LTRfootnote{\lr{mixing parameter for the topology}} می‌گویند،
  نسبت یال‌های درون انجمن به یال‌های بیرون انجمن را کنترل می‌کند
  به این صورت که نسبت یال‌هایی که بین انجمن‌ها قرار دارند به کل یال‌ها، برابر با $\mu_t$
  و در نتیجه نسبت یال‌های که درون انجمن‌ها قرار دارند برابر با $1 - \mu_t$ خواهد بود.
  \item $\mu_w$:
  همانند $\mu_t$ پارامتر دیگری نیز وجود دارد که توزیع وزن‌ها را کنترل می‌کند.
  این پارامتر که به آن  \Muw\LTRfootnote{\lr{mixing parameter for the weights}}  می‌گویند، $\mu_w$ نام دارد و همانند $\mu_t$، برابر است با نسبت مجموع وزن پیوندهای بین انجمن‌ها به مجموع وزن کل پیوندها.
  \item پارامترهای دیگری نیز وجود دارند مانند حداقل و حداکثر اندازه انجمن‌ها، ضریب خوشه‌بندی میانگین و غیره
\end{itemize}
این شبکه‌ها ابتدا در سال ۲۰۰۸ برای حالت بدون وزن معرفی شدند \cite{lancichinetti2008benchmark}. سپس در سال ۲۰۰۹، لانچیکینتی و همکاران در مقالهٔ دیگری حالت وزن‌دار این شبکه‌ها را نیز معرفی کردند \cite{lancichinetti2009benchmarks}. همان‌طور که در این مقاله به تفصیل توضیح داده شده‌است، برای ساخت شبکه‌های \lr{LFR} وزن‌دار، ابتدا یک شبکه \lr{LFR} بدون وزن با استفاده از $\mu_t$ ساخته می‌شود. سپس وزن‌ها به پیوندها طوری تخصیص داده می‌شوند که $s_i^{(internal)} = (1 - \mu_w)s_i$ باشد. در این معادله $s_i$ قدرت گره $i$ است که به معنی مجموع وزن پیوندهاییست که به گره $i$ متصلند و $s_i^{(internal)}$ به معنی مجموع وزن پیوندهاییست که گره $i$ با گره‌های هم‌انجمن خود برقرار کرده است. توضیحات بیشتر برای نحوهٔ انجام این کار در \cite{lancichinetti2009benchmarks} آمده است. برای تولید شبکه‌های استفاده شده در این پژوهش، از نرم‌افزاری که توسط شخص لانچیکینتی آماده شده و بر روی وبگاه وی قرار گرفته است \cite{lfrweb}، استفاده شده‌است.
