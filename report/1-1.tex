\section{مقدمه}
امروزه بسیاری از سامانه‌های اجتماعی، اطلاعاتی و زیستی را می‌توان با استفاده از شبکه‌ها توصیف کرد؛ که در آن، گره‌ها\LTRfootnote{Nodes} معرف افراد هستند و یال‌ها\footnote{در تمامی این گزارش، کلمه‌های یال ($edge$) و پیوند ($link$) معادل هم هستند.} (پیوندهای بین گره‌ها) ارتباط یا تعامل بین گره‌ها را مشخص می‌کنند. شبکه‌های اجتماعی برخط\LTRfootnote{Online} مانند توییتر\LTRfootnote{Twitter}، لینکداین\LTRfootnote{LinkedIn}
و…، شبکه‌های پرسش و پاسخ برخط مانند استک‌ اُوِر فِلو\LTRfootnote{StackOverFlow}، پاسخ‌های یاهو\LTRfootnote{Yahoo Answers}
و…، شبکه‌های ارتباط بین ژن‌ها یا پروتئین‌ها، زنجیره‌ی غذایی، شبکه‌ی ارتباطی فرودگاه‌های کشور، شبکه‌ی زیرساخت اینترنت، شبکه‌ی برق‌رسانی کشور و… از این دسته‌اند.
با توجه به این موضوع، امروزه مطالعه شبکه‌های پیچیده\LTRfootnote{\lr{Complex Networks}} تبدیل به یک نقطه توجه مشترک بین شاخه‌های مختلف علم شده است. تلاش‌های بسیاری در راستای فهم سیر تکاملی شبکه‌ها، ارتباط میان همبندی\LTRfootnote{\lr{topology}} شبکه و عملکرد آن، و ویژگی‌های شبکه انجام شده‌است.
یکی از مسائل علمی مهم مرتبط با تحلیل شبکه‌ها، مسئله‌ایست موسوم به \textit{بازیابی اطلاعات}\LTRfootnote{\lr{Information Retrival}} که هدف آن، به دست آوردن اطلاعات مفید و مورد نیاز، از یک توده‌ی عظیم داده‌هاست. همچنین می‌توان به این موضوع از منظر پیش‌بینی ارتباطات بین افرادِ شبکه، و به صورت گسترده‌تر مسئله‌ی پیوندکاوی\LTRfootnote{\lr{Link Mining}}، نگاه کرد که در این میان، مسئله‌ی \textit{پیش‌بینی پیوند}\LTRfootnote{\lr{Link Prediction}} ، یکی از بنیادی‌ترین مسائل است که سعی دارد تا شانس وجود و یا تشکیل پیوند بین دو گره را، با استفاده از مشاهدات روی پیوند‌ها و ویژگی‌های گره‌ها، تخمین بزند.
  مسئلهٔ پیش‌بینی پیوند را می‌توان در دو بخش عمده دسته‌بندی کرد: دستهٔ اول پیش‌بینی پیوندهایی است که وجود دارند اما هنوز ناشناخته باقی مانده‌اند، مثل شبکه‌های غذایی، شبکهٔ تعاملی پروتئین-پروتئین و شبکه‌های زیستی\LTRfootnote{Biological Networks}؛ دسته‌ی دیگر پیش‌بینی پیوندهایی است که ممکن است در آینده در یک شبکه پویا و در حال تغییر و تکامل، مانند شبکه‌های اجتماعی برخط، ایجاد شوند.
%برای دسته‌ی نخست، از آن‌جای که دانش ما از این شبکه‌ها و ارتباطات بین گره‌ها در آن‌ها، بسیار محدود و جستجوی پیوندها در آزمایشگاه بسیار پرهزینه است، جستجوی کورکورانه‌ی تمام تعاملات ممکن بین گره‌ها به هیچ عنوان مقرون به صرفه نیست. درعوض، پیش‌بینی این که چه پیوندهایی با احتمال بیشتری ممکن است وجود داشته باشند و تمرکز و آزمایش روی آن‌ها، می‌تواند هزینه‌های آزمایش را به شدت کاهش دهد.برای دستهٔ دوم نیز، برای مثال در شبکه های اجتماعی برخط، پیوندهای ناموجود اما بسیار محتمل، می‌توانند به عنوان گزینه‌های جدیدی برای دوستی، به کاربرها پیشنهاد شوند که این امر باعث می‌شود که کاربرها بتوانند دوستی‌ها و رابطه‌های جدید ایجاد کنند و در نتیجه میزان رضایت و استفادهٔ آن‌ها از وبگاه، افزایش یابد.
در بیشتر تحقیقاتی که تاکنون در این حوزه انجام گرفته، وزن پیوندها لحاظ نشده و همهٔ آن‌ها، هم‌وزن در نظر گرفته شده‌اند، اما در بسیاری از شبکه‌ها، پیوندها دارای وزن هستند و میزان ارتباط بین گره‌ها می‌تواند در پیش‌بینی تاثیرگذار باشد. هدف این پژوهش، بررسی تاثیر مشارکت وزن پیوندها در کیفیت پیش‌بینی است. این پژوهش همچنین می‌کوشد تا از ساختار انجمن‌های شبکه‌ها کمک بگیرد تا بتواند به بهبود کارایی روش‌های پیش‌بینی پیوند کمک کند.
  
 
