\section{ساختار پایان‌نامه}

\begin{description}
  \item[فصل \ref{ch:litre}:] در این فصل ابتدا مسئلهٔ پیش‌بینی پیوند به طور رسمی معرفی می‌شود و تعاریف، اصول و مبانی نظری این مسئله مورد بررسی قرار می‌گیرد. سپس روش‌های مختلف پیش‌بینی پیوند دسته‌بندی می‌شوند و هر کدام از این روش‌ها به صورت مختصر معرفی می‌شوند. در این بخش روش‌هایی که در این پژوهش مورد استفاده قرار گرفته‌اند با جزئیات بیشتری بررسی خواهند شد. در ادامه به دلیل استفاده از روش‌های تشخیص انجمن در این پژوهش، مروری اجمالی بر این روش‌ها نیز انجام خواهد گرفت و روش مورد نظر معرفی خواهد شد.
  
  \item[فصل \ref{ch:propo}:] در این فصل ابتدا به بیان مقدمه و پیش‌نیازهای بحث پرداخته می‌شود و سپس روش پیشنهادی این پژوهش معرفی می‌گردد که همان استفاده از اطلاعات انجمن‌ها و پیش‌بینی پیوند در داخل انجمن است. سپس تعریف ریاضی روش بررسی خواهد شد. در ادامه در مورد چگونگی تاثیر وزن یال‌ها در روش پیشنهادی بحث خواهد شد. سپس یک رویکرد عملی متفاوت برای محاسبهٔ تقریبی شاخص‌های پیشنهادی ارائه شده و در مورد نقاط قوت و ضعف آن صحبت خواهد شد. در پایان نیز جمع‌بندی کوتاهی از این فصل ارائه می‌شود.
  
  \item[فصل \ref{ch:exper}:] در این فصل ابتدا مجموعه داده‌های مورد استفاده در این پژوهش معرفی می‌شوند. سپس معیارهای ارزیابی روش‌های پیش‌بینی پیوند به طور کامل مورد بحث قرار می‌گیرند. در نحوه انجام آزمایش‌ها توضیح داده می‌شود و نتایج به دست آمده از آن‌ها در قالب نمودارها و جداول ارائه شده و با توجه به معیارهای مختلف ارزیابی تشریح و تفسیر، و با هم مقایسه می‌شوند.
  
  \item[فصل \ref{ch:concl}:] در این فصل نیز که فصل پایانی این پژوهش است، یک جمع‌بندی از مطالب ارائه‌شده در این پژوهش بیان می‌شود. و در نهایت کارهای آینده‌ای که در راستای این پژوهش می‌توانند مورد توجه قرار گیرند، بحث و بررسی خواهند شد.
  
  
\end{description}
